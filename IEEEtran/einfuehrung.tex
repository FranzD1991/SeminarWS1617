\section{Einf\"uhrung}
Bei der Realisierung eines Softwareentwicklungsprojektes kann eine Kluft zwischen den Software-Architekten auf der einen Seite und den Requirements Engineers auf der anderen Seite bestehen. Software Architekten k\"onnen mit dem Problem konfrontiert sein, dass Anforderungsdokumente nicht ausreichend sind um weitreichende Architekturentscheidungen in Bezug auf den Softwareentwurf zu treffen. Daher ensteht in diesem Fall f\"ur den Software-Architekten h\"aufig der Mehraufwand, dass dieser \"uber weitere Interviews ein pr\"aziseres Bild von der gew\"unschten Software Architektur gewinnen muss \cite{Ros01}. Dies resultiert in Verz\"ogerungen, die sich dann darin wiederspiegeln, dass Termine nicht eingehalten werden k\"onnen. Betreibt der Software Architekt diesen Mehraufwand nicht und trifft eigene Annahmen bez\"uglich der Software Architektur \cite{Ros01} kann dies wiederum zu einer geringeren Akzeptanz des Kunden und im schlimmsten Fall zum Auftragsverlust f\"uhren. Die Requirements Engineers wissen auf der anderen Seite nicht, welche Anforderungen konkret wichtig f\"ur den Software Architekten sind, da ihnen die entsprechende Fachkenntnis fehlt \cite{Ros01}. Somit k\"onnen diese nicht zielf\"uhrend die notwendigen Informationen mit dem Kunden erarbeiten. Ohne diese Informationen ist eine ausreichende Grundlage der Anforderungen f\"ur den Architekturentwurf nicht gegeben. Um diese Kluft zu \"uberbr\"ucken ist es notwendig Verfahren zu ermitteln, die es Requirements Engineers erm\"oglicht die richtigen Informationen einzuholen und die Zusammenarbeit mit den Software Architekten zu optimieren.\\

Im Folgenden wird zun\"achst in der Problemstellung klar hervorgehoben, welche Probleme es in der Zusammenarbeit von Software-Architekten und Requirements Engineers gibt. Nachfolgend werden vier Verfahren untersucht die das Potenzial haben die Zusammenarbeit zwischen Requirements-Engineer und Software-Architekt zu verbessern. Diese vier Verfahren sind:\\

\begin{itemize}
\item Probing wird beschrieben in \ref{probing}
\item CBSP wird beschrieben in \ref{cbsp}
\item COSMOD-RE wird beschrieben in \ref{scgo}
\item ADD 3.0 wird beschrieben in \ref{add3}\\
\end{itemize}

Diese vier Verfahren werden nach einem festgelegten Schema untersucht. Hierbei wird sowohl untersucht, was die Methoden bezwecken wollen, als auch wie die Methoden arbeiten. Auf dieses Schema wird in \ref{unters} näher eingegangen. \\

Nach der Beschreibung der einzelnen Methoden werden diese vergleichend in \ref{auswertung} ausgewertet. Hierf\"ur werden die Methoden zun\"achst unter den eben genannten Punkten verglichen. Danach werden sie in den Kontext der in \ref{problem} genannten Probleme gebracht und hinsichtlich ihrer F\"ahigkeit diese zu l\"osen betrachtet. Es wird \"uberpr\"uft wo diese Verfahren ihre St\"arken und Schw\"achen offenbaren. \\

Abschlie\ss{}end wird \"uberpr\"uft wie diese Verfahren gewinnbringend kombiniert werden oder sich gegenseitig erg\"anzen k\"onnen. \\
