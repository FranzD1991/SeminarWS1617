\subsection{Probing (JEM)}\label{probing}

Das Ziel des Probing ist es, Requirements Engineers mit Fragen auszustatten, die eine Erhebung architekturrelevanter Anforderungen, den ASFRs, erm\"oglichen. Diese Fragen werden Probing Questions (PQ) genannt. \\

\subsubsection{Ziele der Methode}

Die Haupttreiber architekturrelevanter Entscheidungen sind nicht-funktionale (NFR) bzw. qualitative Anforderungen \cite{Ros03}. Diese haben eine explizite Auswirkung auf die zu erstellende Architektur. Bei funktionalen Anforderungen bzw. ASFRs sind diese Auswirkungen meist implizit und m\"ussen zun\"achst durch weitere Interviews mit dem Kunden herausgearbeitet werden. Auch sind hier die f\"ur einen Software-Architekten erforderlichen Informationen nicht immer klar in der Anforderung aufgef\"uhrt \cite{Ros03}. Dies kann zu falschen architektonischen Entscheidungen f\"uhren. Um dieses Problem der nicht klar hervorgehobenen architekturrelevanten Anforderungen anzugehen, besch\"aftigen sich die Probing Questions mit den ASFRs. Requirements Engineers sollen mit Probing Questions ausgestattet werden um zus\"atzliche relevante Fragen stellen zu k\"onnen. Dadurch soll w\"ahrend der Anforderungserhebung eine vollst\"andigere Anforderungsspezifikation erstellt werden, in welcher die ASFRs aussagekr\"aftiger sind \cite{Ros03}. Auch helfen diese den Requirements Engineers, ein genaueres Verst\"andnis f\"ur den Software-Architekten aufzubauen, welche Informationen er f\"ur die Konzeption und Implementierung der Software-Architektur ben\"otigt. \\

\subsubsection{Funktionsweise der Methode}

Da das Probing eine Art Interview-Leitfaden bestehend aus PQs f\"ur die ausf\"uhrlichere Extraktion von ASFRs und keinen methodischen Leitfaden f\"ur die Anwendung dieser zur Verf\"ugung stellt, wird in den folgenden Paragraphen auf die PQs und deren Kategorisierung und nicht auf die Durchf\"uhrung eines Interviews mit diesen eingegangen. \\

\paragraph{Randbedingungen}

Das Probing ben\"otigt, au\ss{}er den mitgebrachten PQs, keine weiteren Artefakte. Allerdings m\"ussen zwei offensichtliche Bedingungen erf\"ullt sein, damit das Probing den Nutzen bringt den es verspricht. Zum einen m\"ussen die PQs oder sogar die PQ-Flows dem Requirements Engineer zur Verf\"ugung gestellt werden. Zum anderen muss dieser auch bereit sein, diese bei der Anforderungsgewinnung zu nutzen. \\

Damit die PQs und PQ-Flows genauer auf ein Projekt zugeschnitten sind und besser auf einen konkreten Anwendungsfall angewandt werden k\"onnen, m\"ussen diese von einem erfahrenen Software-Architekten ausgew\"ahlt, angepasst oder erg\"anzt werden. Daf\"ur ben\"otigt dieser allerdings ausreichend Wissen und Erfahrung in der Dom\"ane und \"uber den Kunden \cite{Ros02}. Mehr dazu unter \ref{probing_model}. \\

\paragraph{Eingabe}

F\"ur die Anwendung stellt das Probing sowohl einen Katalog aus PQs als auch zugeh\"orige PQ-Flows bereit. Damit die PQs m\"oglichst alle Bereiche der ASFRs abdecken wurden letztere, wie in Tabelle \ref{tab:asfr_category_table} zu sehen, in 15 Kategorien unterteilt. Die Kategorien wurden durch die Analyse von 450 FRs aus 30 Anforderungsspezifikationen und durch mehrere Interviews mit erfahrenen Software-Architekten aus verschiedenen Dom\"anen und L\"andern aufgestellt  \cite{Ros02} \cite{Ros03}. Sie bilden einen ersten Ansatz f\"ur die Strukturierung von PQs und sollen in Zukunft verfeinert werden \cite{Ros03}. \\ 

\begin{table}[h] %h, t, b, p : here = genau an dieser Stelle, wenn m\"oglich, top = f\"ur den Seitenanfang, bottom = f\"ur das Seitenende, p = f\"ur eine spezielle Seite mit Tabellen und Abbildungen
\caption{ASFR Kategorien nach \cite{Ros02}}
\centering
\begin{tabular}{|p{3cm}|p{5cm}|}
%\begin{tabular}{|p{2cm}|p{4cm}|p{4cm}|p{6cm}|}%c, l, r, p{SIZEcm}, | : Zentrierter Text, linksb\"undiger Text, rechtsb\"undiger Text, Die Spalte soll SIZE cm breit sein, Das sog. Pipe-Symbol zeichnet einen senkrechten Strich zwischen die beiden Spalten
	\hline
	\textbf{ASFR Kategorie} & \textbf{Beschreibung} \\ %& \textbf{Beispiel ASFR} & \textbf{Beispiel Probing Question(s)} \\
	\hline
  	Audit Trail & Auditierung der Systemausf\"uhrung erm\"oglichen \\%& Das System muss jede \"anderung an Kundendatens\"atzen f\"ur Auditzwecke erfassen. & - M\"ussen vor und nach Snapshots der Datenbank Tabellen gespeichert werden? \newline - Welche Art von Logs m\"ussen gef\"uhrt werden? \\
	\hline
	Batch-Verarbeitung (Batch Processing) & Batch-Verarbeitung erm\"oglichen \\
	\hline
	Lokalisierung & Unterst\"utzung f\"ur mehrere Sprachen anbieten \\
	\hline
	Gesch\"aftsprozess ``Status'' Meldungen (Kommunikation) & Unterst\"utzung f\"ur Kommunikation anbieten \\
	\hline
	Datenbezogene Dialoge & Unterst\"utzung verschiedener Dateneingabemechanismen \\
	\hline
	Bezahlung (Payment) & Finanzielle Transaktionen erm\"oglichen \\
	\hline
	Druck & Unterst\"utzung zum Drucken von Dokumenten anbieten \\
	\hline
	Berichte (Report) & Die Generierung von Berichten erm\"oglichen \\
	\hline
	Suche & Anbieten von Suchfunktionen \\
	\hline
	Drittanbieter-Interaktion & Interaktion mit Drittanbieter-Komponenten erm\"oglichen \\
	\hline
	Arbeitsfluss (Workflow) & Bereitstellung von Funktionen um Arbeitselemente zu verschieben, Gutachten und Genehmigungen erm\"oglichen\\
	\hline
	Online-Hilfe & Online-Hilfe erm\"oglichen \\
	\hline
	Lizensierung & Dienste f\"ur die Beschaffung, Installation und Beobachtung der Lizenzbenutzung erm\"oglichen \\
	\hline
	Analyse von Benutzerverhalten & Sammeln, Analysieren und Aggregieren von Benutzerverhaltensdaten erm\"oglichen \\
	\hline
	Speichermechanismen & Bereitstellung von automatischen Mechanismen zur Speicherung und Konvertierung von Dokumenten \\
	\hline
\end{tabular}
\label{tab:asfr_category_table}
\end{table}

Zu jeder Kategorie von ASFRs sind sowohl mehrere PQs als auch beispielhafte ASFRs f\"ur ein besseres Verst\"andnis der Kategorie notiert. F\"ur die vierte Kategorie (Kommunikation) lautet eine Beispiel-ASFR \textit{Ein Arbeitsfluss ist erforderlich, um Benachrichtigungen an die Versicherer zu senden}. M\"ogliche PQs hier sind \textit{Soll die Kommunikation unidirektional oder bidirektional verlaufen?} oder \textit{Soll die Kommunikation vom Kunden konfigurierbar sein?} \cite{Ros03}. F\"ur die Kategorie Suche wurden PQs wie \textit{Soll die Suche schl\"usselwort- oder filter-basiert sein?} oder \textit{Ist eine semantische Suche notwendig?} aufgenommen \cite{Ros03}. \\

Des Weiteren sind die aufgestellten PQs in sechs Typen unterteilt \cite{Ros02}: \\

\begin{itemize}
\item[1.] \textit{Gesch\"aftsregeln im Unternehmen des Kunden:} Einige PQs beziehen sich auf Einschr\"ankungen, welche durch Unternehmensregeln gegeben sind, so beispielsweise \textit{Wie lange muss ein Audit-Trail gespeichert werden?} oder \textit{Welche Informationen m\"ochte das Unternehmen im Audit-Trail sehen?} \cite{Ros02}.
\item[2.] \textit{Strategische Technologieentscheidungen:} PQs von diesem Typ beziehen sich auf die architektonischen Auswirkungen, welche durch Investitionsentscheidungen des Unternehmens des Kunden begr\"undet sind \cite{Ros02}.
\item[3.] \textit{Nicht-funktionale Aspekte bezogen auf die ASFRs:} Viele PQs behandeln relevante nicht-funktionale Aspekte, die bei einer architekturrelevanten Entscheidung ber\"ucksichtigt werden m\"ussen. Zum Beispiel beziehen sich folgende PQs auf Skalierbarkeit und Sicherheit: \textit{Wie gro\ss{} ist die Menge an Daten?}, \textit{Wie viele Haupttabellen werden f\"ur die Auditierung ben\"otogt?} oder \textit{Welches Sicherheitslevel wird bei sensitiven Daten ben\"otigt?} \cite{Ros02}.
\item[4.] \textit{Einhaltung gesetzlicher Vorschriften:} PQs, welche von gesetzlichen Vorschriften abh\"angen haben einen gro\ss{}en Einfluss auf architektonische Entscheidungen. Wenn in einem Land beispielsweise bestimmte Inhalte von Gesetzeswegen verboten sind, dann hat diese Einschr\"ankung einen gro\ss{}en Einfluss auf die zu konzipierende Software-Architektur \cite{Ros02}.
\item[5.] \textit{Projektkontext:} Hierzu geh\"oren PQs, die kontextuelle Parameter des Projekts betreffen und Auswirkungen auf die Software-Architektur haben. Diese beinhalten PQs, welche die funktionale Passform, einen Gesch\"aftsfall, Kosten, den Zeitrahmen, die zur Verf\"ugung stehenden Technologien, F\"ahigkeiten auf Kunden- und Verk\"auferseite, Unternehmenskultur, Kundenmentalit\"at oder sogar strategische Anforderungen betreffen \cite{Ros02}.
\item[6.] \textit{Zusammengesetzte Wirkung auf zwei oder mehr ASFRs:} Eine PQ, welche Antworten zu mehr als einer ASFR Kategorie liefert, geh\"ort zu diesem Typ. Eine Beispiel-PQ f\"ur die Kategorien Gesch\"aftsprozess  ``Status'' Meldungen und Drittanbieter-Interaktion lautet: \textit{Was f\"ur eine Best\"atigung m\"ochten sie vom Drittanbieter-Gateway bei erfolgreicher Zahlung empfangen?} \cite{Ros02} \\
\end{itemize}

Diese Typisierung zeigt die Abh\"angigkeiten von ASFRs zu anderen ASFRs, NFRs oder anderen Strukturen im Projekt oder Unternehmen auf \cite{Ros02}. Dies erm\"oglicht, auf diese zu reagieren und daraus resultierenden Problemen rechtzeitig vorzubeugen. \\

Zudem wurden f\"ur f\"unf der 15 ASFR-Kategorien ganze PQ-Flows erstellt. Die f\"unf Kategorien sind \textit{Audit Trail}, \textit{Batch-Verarbeitung}, \textit{Gesch\"aftsprozess ``Status'' Meldungen (Kommunikation)}, \textit{Berichte} und \textit{Arbeitsfluss}. Die Idee hier ist, das architektonische Wissen, welches den PQs innewohnt \cite{Ros02}, in einer Wissensdatenbank aus PQ-Flows zu sammeln, \"ahnlich wie die Design Patterns nach Gamma et al. \cite{Ros03} um eine verbesserte und vollst\"andigere Spezifikation zu erhalten. \\

\begin{figure}[h]
	\centering
	\includegraphics[scale=0.45]{pqflow_communication.png} 
	\caption{PQ-Flow f\"ur Gesch\"aftsprozess ``Status'' Meldungen (Kommunikation) \cite{Ros01}}\label{fig_pqflow_communication}
\end{figure}

Ein PQ-Flow ist eine Art Fluss- oder Ablaufdiagramm aus PQs. Hier sind mehrere PQs so angeordnet, dass die Reihenfolge, in welcher sie im Interview benutzt werden eine Rolle spielt. Dadurch soll dem Requirements Engineer eine bessere Hilfestellung geben werden, da hier einige Fragen abh\"angig von den Antworten vorhergegangener Fragen irrelevant werden k\"onnen. So ist es logisch sinnvoll die PQ \textit{Welche Rollen (Stakeholder) d\"urfen das Privileg erhalten, diese Warnungen zu konfigurieren?} (PQ 4) aus Abbildung \ref{fig_pqflow_communication} nur zu stellen, wenn die vorherige PQ (PQ 3) mit einem Ja beantwortet wurde. \\

\paragraph{Vorgehensmodell} \label{probing_model}

Die bereits existierenden PQs und PQ-Flows k\"onnen ohne weitere \"Anderungen dem Requirements Engineer an die Hand gegeben werden, da diese bereits in der Praxis erfolgreich getestet wurden und laut \cite{Ros01}, \cite{Ros02} und \cite{Ros03} einen Mehrwert bieten. Allerdings bieten diese einen relativ allgemeinen Ansatz und sind nicht auf ein konkretes Projekt einer bestimmten Dom\"ane zugeschnitten. 

Alternativ k\"onnen die PQs mit zus\"atzlichem Wissen des Software-Architekten eines konkreten Projektes angepasst oder erg\"anzt werden, wenn ihm bestimmte Aspekte f\"ur sein Projekt in den PQs nicht ausreichend abgedeckt erscheinen \cite{Ros02}. Hierzu ben\"otigt dieser jedoch Erfahrung und Wissen in folgenden Bereichen \cite{Ros02}: \\

\begin{itemize}
\item[1.] Detailliertes Dom\"anen-Wissen
\item[2.] Kenntnisse \"uber das Gesch\"aftsfeld des Kunden sowie \"uber den Projektzyklus im Unternehmen des Software-Architekten
\item[3.] Wissen \"uber das Unternehmen des Kunden \\
\end{itemize}

F\"ur diese Anpassungen gilt allerdings: ``Erfahrung spielt eine signifikante Rolle'' \cite{Ros02}. \\

\paragraph{Ausgabe}

Nachdem die PQs vom Requirements Engineer bei der Anforderungserhebung benutzt wurden erh\"alt dieser daraufhin eine vollst\"andige Spezifikation mit allen vom Software-Architekten, ben\"otigten ASFRs. \\

%\textbf{Paper Cites:} \\
%(\cite{Ros01}) Probing for Requirements Knowledge to Simulate Architectural Thinking \\
%(\cite{Ros02}) What you see is what you get: Understanding Architecturally Significant Functional Requirements \\
%(\cite{Ros03}) Identifying Architecturally Significant Functional Requirements \\