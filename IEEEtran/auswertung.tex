\section{Auswertung der Methoden}\label{auswertung}
In den vorhergehenden Kapiteln sind verschiedene Methoden vorgestellt worden, die das Potenzial haben, das Zusammenwirken von Requirements Engineer und Software-Architekt zu verbessern und dadurch eine verbesserte Software-Architektur zu generieren. Nun gilt es jedoch zu pr\"ufen in wie weit dieses Potenzial zutrifft. Eignet sich eine Methode besser als die anderen? Wie sieht es aus, wenn die Methoden miteinander verglichen werden? Erg\"anzen sich die Methoden wom\"oglich? Diese Fragen gilt es in der Auswertung zu beantworten.\\

Hierf\"ur werden zun\"achst in tabellarischer Form die Kernattribute der besprochenen Methoden dargestellt. Danach werden die Methoden in Bezug zu den Problemen P1-P7 gebracht und es wird mittels eines Ratings \"uberpr\"uft, wie geeignet die Methode ist, um das Problem zu bew\"altigen. Danach werden die Methoden gegen\"ubergestellt, wodurch \"uberpr\"uft werden soll, wo sich die Methoden potenziell erg\"anzen, bzw. an welchen Stellen die eine Methode besser geeignet ist als die andere.\\ 

\subsection{Vergleich}
In der Tabelle \ref{tab:method_intro} sind die Kernattribute der Methoden dargestellt.\\

\subsubsection{Eignung der Methoden zur L\"osung der Probleme}
In der Tabelle \ref{tab:method_rating} sind die Methoden und ihr Rating in Bezug zu den Problemen dargestellt.\\

\begin{table}[h] %h, t, b, p : here = genau an dieser Stelle, wenn m\"oglich, top = f\"ur den Seitenanfang, bottom = f\"ur das Seitenende, p = f\"ur eine spezielle Seite mit Tabellen und Abbildungen
\caption{Rating der Methoden}
\centering
\begin{tabular}{|l|c|c|c|c|}
\hline 
• & \textbf{Probing} & \textbf{CBSP} & \textbf{COSMOD-RE} & \textbf{ADD 3.0} \\ 
\hline 
\textbf{P1} & 2 & 3 & 3 & 1 \\ 
\hline 
\textbf{P2} & 2 & 3 & 3 & 1 \\ 
\hline 
\textbf{P3} & 2 & 3 & 2 & 1 \\ 
\hline 
\textbf{P4} & 1 & 3 & 1 & 3 \\ 
\hline 
\textbf{P5} & 3 & 2 & 3 & 2 \\ 
\hline 
\textbf{P6} & 3 & 3 & 1 & 2 \\ 
\hline 
\textbf{P7} & 1 & 3 & 3 & 1 \\ 
\hline 
\hline 
Rating & 14 & 20 & 16 & 11 \\ 
\hline 
\end{tabular} 
\label{tab:method_rating}
\end{table}

Eine 1 steht hierbei daf\"ur, dass eine Methode nicht geeignet ist, das Problem zu l\"osen. Die 2 steht daf\"ur, dass eine Methode teilweise daf\"ur geeignet sein kann, das Problem zu l\"osen und eine 3 steht daf\"ur, dass eine Methode gut daf\"ur geeignet ist, das Problem zu l\"osen.\\

\paragraph{Probing}
Begründung warum die werte in der tabelle so sind wie sie sind. \\

\paragraph{CBSP}
Begründung warum die werte in der tabelle so sind wie sie sind. \\


\paragraph{COSMOD-RE}
Begründung warum die werte in der tabelle so sind wie sie sind. \\

\paragraph{ADD 3.0}
Begründung warum die werte in der tabelle so sind wie sie sind. \\


\subsubsection{Gegen\"uberstellung der Methoden}

\paragraph{Vergleich zwischen CBSP und Probing}
Vor und Nachteile im Bezug auf: Parallelität, Probleme, (Zeit, Projektgröße etc.)\\

\paragraph{Vergleich zwischen CBSP und  COSMOD-RE}
Vor und Nachteile im Bezug auf: Parallelität, Probleme, (Zeit, Projektgröße etc.)\\

\paragraph{Vergleich zwischen CBSP und ADD 3.0}
Vor und Nachteile im Bezug auf: Parallelität, Probleme, (Zeit, Projektgröße etc.)\\

\paragraph{Vergleich zwischen ADD 3.0 und Probing}
Vor und Nachteile im Bezug auf: Parallelität, Probleme, (Zeit, Projektgröße etc.)\\

\paragraph{Vergleich zwischen COSMOD-RE und ADD 3.0}
Im Vergleich von COSMOD-RE und ADD 3.0 gibt es eine Vielzahl von Auffälligkeiten. So haben sowohl ADD 3.0 als auch COSMOD-RE als ein Ziel den Entwurf einer Software-Architektur. Zusätzlich hat COSMOD-RE jedoch noch weitere Ziele, wie die Gewinnung von Anforderungen. Um COSMOD-RE anzuwenden müssen einige Randbedingungen geklärt werden, wie beispielsweise die Definition der Grenzen zwischen verschiedenen Ebenen der Abstraktionshierarchie oder eine Möglichkeit der Verknüpfung von Architektur- und Anforderungsmodellen. Außerdem muss eine System-Vision vorhanden sein. Ähnlich umfangreich sind die Randbedingungen von ADD 3.0. Hier wird vorausgesetzt, dass die Anforderungserhebung abgeschlossen ist und die Projekttreiber vorhanden sind. Besonders die Qualitätsattribute sollten vernünftig erhoben sein. Die ersten großen Unterschiede ergeben sich bei der Betrachtung der Eingaben. COSMOD-RE benötigt als Eingabe lediglich die System-Vision, während ADD 3.0 eine Menge von Projekttreibern benötigt. Die große Menge an ausführlichen Projekttreibern sind ein Nachteil an ADD 3.0, den COSMOD-RE nicht hat. Wenn es darum geht, einen schnellen Einstieg in die Methode zu erlangen, hat COSMOD-RE hier einen Vorteil. Ein Nachteil von COSMOD-RE offenbart sich jedoch, wenn die Ausgaben betrachtet werden. Während COSMOD-RE hier lediglich grobe Architekturartefakte liefert, kann ADD 3.0 eine verfeinerte Software-Architektur generieren. Dies bedeutet wenn der Mehraufwand zu beginn betrieben wird, ist es möglich eine bessere Ausgabe zu produzieren.

Bei COSMOD-RE wird als Eingabe lediglich die Systemvision benötigt.
Vor und Nachteile im Bezug auf: Parallelität, Probleme, (Zeit, Projektgröße etc.)\\

\paragraph{Vergleich zwischen Probing und  COSMOD-RE}
Vor und Nachteile im Bezug auf: Parallelität, Probleme, (Zeit, Projektgröße etc.)\\

\subsection{Bewertung}
Unter der Ber\"ucksichtigung der genannten Probleme haben einige Methoden ein h\"oheres Rating erzielt als andere. Insgesamt betrachtet hat CBSP mit 20 Punkten das h\"ochste Rating, wenn als Grundlage die Menge der behandelten Probleme gegeben ist. COSMOD-RE hat mit 16 Punkten das zweith\"ochste Rating, Probing mit 14 das dritth\"ochste und ADD 3.0 mit 11 Punkten das niedrigste Rating. Dies bedeutet, CBSP und COSMOD-RE liefern bessere L\"osungsans\"atze f\"ur die meisten Probleme. Hierbei ist allerdings zu beachten, dass CBSP und COSMOD-RE einen sehr gro\ss{}en Bereich des in Abbildung \ref{methoden} skizzierten Prozesses abdecken, w\"ahrend Probing und ADD 3.0 nur einen Teil davon behandeln. Zudem bieten sowohl Probing als auch ADD 3.0 haupts\"achlich nur f\"ur einen Teil der unter \ref{problem} aufgestellten Probleme einen L\"osungsansatz an. \\

Bei der Betrachtung der Kompabilit\"at der Methoden f\"allt auf, dass sowohl CBSP als auch COSMOD-RE um Probing erweiterbar sind. In Kombination mit Probing w\"urde CBSP ein Rating von 21 erzielen. Bei COSMOD-RE w\"urde sich das Rating bei der Kombination mit Probing und ADD 3.0 auf 20 erh\"ohen, womit alle Probleme au\ss{}er dem fehlenden Know-How gut zu l\"osen w\"aren. Eine Kombination aus Probing und ADD 3.0 w\"urde lediglich ein Rating von 16 erzeugen, was bei der Kombination von Methoden das schw\"achste Ergebnis liefert. Andere Kombinationen wie beispielsweise CBSP und COSMOD-RE oder CBSP und ADD 3.0 sind aufgrund von Konflikten in der methodischen Ausf\"uhrung nicht m\"oglich. \\
