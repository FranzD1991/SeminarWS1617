\subsection{Probing (JEM)}

\subsubsection{Beschreibung}

TODO: ASFRs früher erklären??! \\

Da die beiden Bereiche des Requirements-Engineering und der Software-Architektur ein enormes Maß an Wissen und Fähigkeiten erfordern werden diese in der Regel von verschiedenen Teams betreut. Wie bereits genannt, entstehen hier häufig Probleme durch fehlendes technisches Know-How über Software-Architektur spezifische Aspekte bei den Requirements-Engineers. Auch können diese oft nicht zwischen funktionalen Anforderungen (FR) und architekturspezifischen funktionalen Anforderungen (ASFR) unterscheiden. ASFRs sind FRs welche kritisch, sehr risikobehaftet, volatil und bei Änderungen ein aufwendiges oder teures Refactoring mit sich bringen würden oder einen anderweitig großen Impakt auf die zu konzipierende Software-Architektur hätten. Durch das fehlende Know-How entstehen unvollständige Anforderungs-Artefakte, in welchen wesentliche ASFRs für den Software-Architekten fehlen. \\

Das Ziel des Probing ist es, Requirements Engineers mit Fragen auszustatten die eine Erhebung architekturrelevanter Anforderungen, den ASFRs, ermöglichen. Diese Fragen werden Probing Questions (PQ) genannt. Software-Architekten stellen PQs meist intuitiv auf Erfahrung basierend. \\

TODO: PQ Kategorien und Typen aus (2). \\

TODO: PQ-Flow aus (1). \\


% Kunden nehmen oft an, das Requirements Engineer die implizit ausgedrückten Anforderungen verstehen und aufführen. Das liegt daran, dass manche Aspekte für sie so selbstverständlich in ihrer täglichen Arbeit sind, dass sie nicht daran denken diese explizit aufzuführen. 


\textbf{Paper so far:} \\
(1) Probing for Requirements Knowledge to Simulate Architectural Thinking \\
(2) What you see is what you get: Understanding Architecturally Significant Functional Requirements \\
(3) Identifying Architecturally Significant Functional Requirements \\

\subsubsection{Bewertung}

Bewertung (Chancen / Grenzen) (Was bleibt offen) und Ausblick \\

To be continued ... \\

TODO: ae, ue, und oe im Text ersetzen! \\
