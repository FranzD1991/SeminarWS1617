\section{Problemstellung}
Während der Zusammenarbeit zwischen Requirements Engineer und Software Architekt können vielfältige Probleme auftreten. Bei der Untersuchung möglicher Probleme lässt sich eine Kategorisierung dieser vornehmen. So sind einige Probleme bedingt durch den Menschen selbst, während andere Probleme sich aus den Anforderungen ergeben. Im folgenden werden die kategorisierten Probleme genauer ausgeführt. \\

\subsection{Durch den Menschen bedingt}
Bei der Betrachtung der direkt durch den Menschen bedingten Probleme fallen folgende auf:\\

\begin{itemize}
\item[P1:] \textit{Schlechte Kommunikation} 
\item[P2:] \textit{Konkurrierende Interessen}
\item[P3:] \textit{Fehlendes Know-How} \\
\end{itemize}

\subsubsection{Schlechte Kommunikation}
Eine schlechte Kommunikation ist in diesem Zusammenhang als eine Kommunikation zu sehen, bei der wesentliche Intentionen und Informationen nicht übermittelt werden. Ein wichtiger Aspekt schlechter Kommunikation ist die Distanz zwischen den Akteuren innerhalb eines Projektes.\\

Nach \cite{Bja13} kann vor allem Distanz ein Grund für schlechte Kommunikation sein. Distanzen in der Kommunikation zwischen Requirements Engineer und Software Architekt können die Koordination reduzieren, was wiederum Projektverzögerungen hervorrufen kann, sowie das nicht erfüllen von Kundenwünschen. Distanzen werden nach \cite{Bja13} vor allem als räumliche Distanzen bei global agierenden Projektteams gesehen. Bei räumlichen Distanzen können nach \cite{Her01} beispielsweise Begleitprobleme wie das Beherrschen einer Sprache aufkommen. So kann das schlechte Beherrschen einer Sprache zu Missverständnissen führen und Fehler in den Anforderungen hervorrufen.  Zusätzlich können nach \cite{Bja13} soziale, kulturelle oder zeitliche Distanzen hinzukommen.\\

Bei nicht ausreichendem Kommunikationsfluss zwischen Requirements Engineer und Software Architekt kann das Problem aufkommen, dass der Requirements Engineer, während der Anforderungsgewinnnung, keine Rücksprache mit dem Software Architekten hält \cite{7}. Ohne ausreichende Rücksprache kann beispielsweise eine Beeinträchtigung der Qualität der architekturrelevanten Anforderungen auftreten  \cite{7}. Dies kann von fehlenden bis fehlerhaft Anforderungen reichen. Ein weiteres mögliches Problem ist, dass der Software Architekt dem Requirements Engineer nicht ausreichend vermittelt, welche Informationen er für einen gültigen Architekturentwurf benötigt  \cite{7}. Aus diesen Kommunikationsproblemen können Missverständnisse in den Verantwortlichkeiten folgen  \cite{7}. Auch hier ist eine mögliche Folge die negative Beeinflussung der Qualität der Architekturanforderungen.\\

\subsubsection{Konkurrierende Interessen}
Unter konkurrierenden Interessen ist zu verstehen, dass die Stakeholder des Entwicklungsprozesses der Software-Architektur in der Projektarbeit verschiedene Interessen verfolgen, die miteinander in Konflikt stehen. Die Stakeholder haben zwar eine gemeinsame Vision von dem fertigen Produkt, verfolgen aufgrund variierender Interessen jedoch eine unterschiedliche Art der Zielerreichung \cite{5}. Weiter kann es sein, dass Stakeholder durch ihre unterschiedlichen Sichten auf das Projekt auf verschiedene Arten mit diesem interagieren. Dies kann wiederum mit den Projektzielen konkurrieren. Der Requirements Engineer vertritt hierbei eher die Sicht des Kunden, während der Software Architekt, den Standpunkt des Entwicklers vertritt \cite{4}. Ein Problem kann ebenfalls auftreten, wenn Stakeholder verschiedene Tools, Prozesse und Arbeitsweisen nutzen um das Projektziel zu erreichen. Wenn hier die Kompromissbereitschaft fehlt kann dies die Arbeit erschweren \cite{7}. \\

\subsubsection{Fehlendes Know-How}
Fehlendes Know-How kann zu diversen Problemen führen. Dabei kann hier zwischen zwei Ebenen unterschieden werden. Einerseits können Probleme aus fehlendem Know-How über die Methodik der jeweils anderen Rolle entstehen \cite{5}\cite{7}. Hierbei können unter anderem Probleme bei der Planung und Umsetzung eines Projektes resultieren, da hierdurch potenziell relevante Vorgehensweisen oder Methoden nicht berücksichtigt werden können. Andererseits können Probleme aus inhaltlich fehlendem Know-How entstehen \cite{5}. So kann es dem Requirements-Engineer zum Beispiel schwer fallen die Schwerpunkte so zu setzen, dass der Software-Architekt die Informationen erhält, die er benötigt \cite{9}. Dies wird dann zum Problem wenn der Requirements-Engineer nicht weiß welche Informationen der Software-Architekt für die Umsetzung benötigt. Ferner können Missverständnisse auftreten, wenn der Software-Architekt bei fachspezifischen Begriffen ein anderes Verständnis hat als der Requirements-Engineer \cite{7}. \\

\subsection{Qualit\"at der Anforderungen}
Mit der Untersuchung der Probleme, die sich auf die Qualität der Anforderungen beziehen, fallen folgende auf:\\

\begin{itemize}
\item[P4:] \textit{Zu restriktive / detaillierte architekturrelevante Anforderungen}
\item[P5:] \textit{Ungenaue/ fehlende architekturrelevante Anforderungen}
\item[P6:] \textit{Nicht klar hervorgehobene architekturrelevante Anforderungen}
\item[P7:] \textit{Wechselwirkungen zwischen architekturrelevanten Anforderungen}\\
\end{itemize}

\subsubsection{Zu restriktive / detaillierte architekturrelevante Anforderungen}
Vor allem bei sehr großen Projekten kann es passieren, dass es tausende von Anforderungen, die für die Architektur relevant sind, gibt \cite{5}. Es besteht eine hohe Wahrscheinlichkeit, dass in einem solchen Fall viele Anforderungen Widersprüche erzeugen und den Lösungsraum zu sehr einschränken. Hier können restriktive oder detaillierte Anforderungen die Arbeit von Requirements-Engineer und Software-Architekt unnötig einschränken und dadurch erschweren. \\

\subsubsection{Ungenaue / Fehlende architekturrelevante Anforderungen}
In der Anforderungsgewinnung kann es passieren, dass wichtige architekturrelevante Anforderungen nicht erhoben werden, oder wichtige Details fehlen \cite{6}\cite{8}. Dies kann jedoch weitreichende Auswirkungen auf die Software-Architektur haben, da dem Software-Architekt notwendige Informationen bei der Konzeption fehlen. Dieser sieht sich dann gezwungen entweder eine Software-Architektur frei zu entwerfen oder zusätzliche klärende Gespräche mit dem Kunden zu führen. Auch hier kann es passieren, dass bei besonders großen Projekten mit mehreren tausenden Anforderungen, wichtige nicht erhoben werden \cite{5}. Vor allem im Kontext der agilen Softwareentwicklung bedingt ein iteratives Vorgehen, dass manche architekturrelevanten Anforderungen erst während der Modellierung oder sogar Implementierung der Architektur gewonnen werden \cite{5}. \\

\subsubsection{Nicht klar hervorgehobene architekturrelevante Anforderungen}
Eines der Hauptprobleme architekturrelevanter Anforderung liegt darin, dass es üblicherweise schwer ist diese zu identifizieren und zu spezifizieren \cite{5}. Werden architekturrelevante Anforderungen für den Software-Architekten nicht hervorgehoben, kann es passieren, dass dieser wichtige Anforderungen erst spät, wenn überhaupt wahrnimmt \cite{9}. Dies kann zur Folge haben, dass wichtige architekturrelevante Entscheidungen nicht rechtzeitig getroffen werden können und zu einem Mehraufwand zu einem späteren Zeitpunkt führen \cite{6}. Zusätzlich liegt ein Problem in der fehlenden Ausführung der Auswirkungen von architekturrelevanten Anforderungen auf die spätere Software-Architektur \cite{9}. So kann hieraus folgen, dass wichtige Projektziele nicht erfüllt werden, da der Zeitrahmen des Projektes überschritten wird. \\

\subsubsection{Wechselwirkung zwischen Anforderungen}
Unter Wechselwirkung zwischen Anforderung ist zu verstehen, dass Anforderungen teilweise voneinander abhängig sein können. Diese Abhängigkeit kann auf verschiedene Arten und Weisen erfolgen. Ist beispielsweise ein Ansatz gegeben, bei dem mehrere Abstraktionsstufen vorhanden sind kann es passieren dass wichtige Abhängigkeiten, die über mehrere Abstraktionsstufen hinweg reichen nicht berücksichtigt werden oder weitere Probleme verursachen \cite{1}. Änderungen von Anforderungen auf verschiedenen Hierarchieebenen treffen kontinuierlich ein und können somit Einfluss auf die Architektur haben und erheblichen Änderungsbedarf verursachen \cite{3}. Auch ist es schwierig die Konsistenz und Nachverfolgbarkeit der Projektion von Anforderungen auf die Software-Architektur zu gewährleisten, da eine Anforderung mehrere architekturrelevante Artefakte betreffen kann. Zusätzlich kann ein architekturrelevantes Artefakt mehrere Anforderungen betreffen \cite{5}.\\

Neben den aufgeführten Problemen gibt es weitere, die hier nicht näher behandelt werden. Darunter fällt zum Beispiel ein phasenbezogenes Requirements Engineering.
\\