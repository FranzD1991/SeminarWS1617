\section{Problemstellung}
Während der Zusammenarbeit zwischen Requirements Engineer und Software Architekt können vielfältige Probleme auftreten. Bei der Untersuchung möglicher Probleme lässt sich eine Kategorisierung dieser vornehmen. So sind einige Probleme \textcolor{red}{bedingt durch das Personal}, während andere Probleme sich aus der Qualität der Anforderungen ergeben. Im folgenden werden die kategorisierten Probleme genauer ausgeführt.

\subsection{Personal}
Bei der Betrachtung der direkt durch das Personal bedingten Probleme fallen folgende besonders auf:
\begin{itemize} 
\item Schlechte Kommunikation
\item Konkurrierende Interessen
\item Fehlendes Know-How \\
\end{itemize}

\subsubsection{Schlechte Kommunikation}
Die Probleme im Bezug auf eine schlechte Kommunikation sind als grundsätzliches Problem in der Zusammenarbeit zwischen mehreren Personen zu sehen. In diesem Zusammenhang lassen sie sich in zwei Klassen aufteilten. Der ersten Klasse lassen sich Probleme zuordnen, bei denen die Größe des Kommunikationsflusses zwischen Requirements Engineer und Software Architekt nicht ausreichend ist. Der zweiten Klasse werden die Probleme zugeordnet, die aus einem beidseitigen Monolog entstehen. Unter einem beidseitigen Monolog ist hierbei zu verstehen, dass in der Kommunikation zwischen Requirements Engineer und Software Architekt kein richtiger Dialog stattfindet, sondern lediglich Informationen und Handlungsanweisungen ausgetauscht werden. <REWORK>\\

Bei nicht ausreichendem Kommunikationsfluss zwischen Requirements Engineer und Software Architekt kann das Problem aufkommen, dass der Requirements Engineer, während der Anforderungsgewinnnung, keine Rücksprache mit dem Software Architekten hält. Ohne ausreichende Rücksprache kann beispielsweise eine Beeinträchtigung der Qualität der architekturrelevanten Anforderungen auftreten. Dies kann von fehlenden bis fehlerhaft Anforderungen reichen. Der Verursacher dieses Problems ist am ehesten der Requirements Engineer. Ein weiteres mögliches Problem ist, dass der Software Architekt dem Requirements Engineer nicht ausreichend vermittelt, welche Informationen er für einen gültigen Architekturentwurf benötigt. Auch hier ist eine mögliche Folge die negative Beeinflussung der Qualität der Architekturanforderungen.\\

In der zweiten Klasse werden Probleme gruppiert, bei denen die Gesprächspartner keinen zielführenden Dialog führen, d.h. aneinander vorbei reden. Dies kann kann der Fall sein, wenn zwei Gesprächspartner sich nicht gegenseitig zuhören oder aber die besprochenen Inhalte anschließend nicht berücksichtigen. Wenn ein Software Architekt einem Requirements Engineer zum Beispiel nicht zuhört, kann es passieren, dass Anweisungen missverstanden werden und die konzipierte Software Architektur nicht den Wünschen des Kunden entspricht. Ein weiteres Problem ist, dass der Software Architekt die Vorgaben des Requirements Engineer ignoriert und die besprochenen Inhalte nicht berücksichtigt. \\

\subsubsection{Konkurrierende Interessen}

TODO \\

\subsubsection{Fehlendes Know-How}

TODO \\

\subsection{Qualit\"at der Anforderungen}

Mit der Untersuchung der Probleme, die sich auf die Qualität der Anforderungen beziehen, fallen folgende auf:
\begin{itemize}
\item Zu restriktive Anforderungen
\item Fehlende architekturrelevante Anforderungen
\item Ungenaue / sich wiedersprechende architekturrelevante Anforderungen
\item Nicht klar hervorgehobene architekturrelevante Anforderungen\\
\end{itemize}

\subsubsection{Zu restriktive Anforderungen}

TODO \\

\subsubsection{Fehlende architekturrelevante Anforderungen}

TODO \\

\subsubsection{Ungenaue / sich wiedersprechende architekturrelevante Anforderungen}

TODO \\

\subsubsection{Nicht klar hervorgehobene architekturrelevante Anforderungen}

TODO \\

Neben den aufgeführten Problemen gibt es weitere, die hier nicht näher behandelt werden. Darunter fiele zum Beispiel ein phasenbezogene Requirements Engineering.
