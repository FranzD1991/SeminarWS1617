\subsection{Ziel- und Szenario-basierte Änsätze}
\subsubsection{Beschreibung}
In der Anforderungserhebung soll es möglich sein Anforderungen in einer Form zu erheben, die es Software Architekten einfacher macht, den Architekturentwurf zu konzipieren. Um dies zu realisieren bietet sich eine Kombination aus Ziel-basierten Ansätzen und Szenario-basierten Ansätzen an. Die Kombination ist deswegen von Relevanz, weil ein Ansatz allein nicht ausreichen kann um die Anforderungen in angemessener Weise zu erheben.
\subsubsection{Ziel-basierte Ansätze}
Ziel-basierte Ansätze zielen vorrangig darauf ab, ein umfassendes Verständnis der Wünsche und Ziele der Stakeholder sowie auf die zu erzielenden Auswirkungen auf die Systemumgebung ab (Silkora Referenz S.18). Dies bedeutet, dass es bei Ziel-basierten Ansätzen vor allem darauf ankommt, zu verstehen, welche Vision der Stakeholder von dem Zukünftigen System hat. Bei der Erfassung dieser Vision ist ein natürlichsprachlicher Ansatz fehlerbehaftet, da hier sehr aufwändige manuelle Konsistenzprüfungen notwendig wären. Deswegen bieten sich hier vor allem Modell-basierte Ansätze an.\\
Ein gutes Beispiel für ein Modell-basierten Ansatz ist der KAOS-Ansatz. Dieser Ansatz bietet den Vorteil, dass er mit wenigen präzise formulierten Modellierungsobjekten auskommt. Dies ist deswegen ein Vorteil, weil so kein besonders tief reichendes Fachwissen notwendig ist um das Modell zu interpretieren. Ferner ist der Ansatz für die Konzeption softwareintensiver eingebetteter Systeme geeignet, was eine verzahnte Entwicklung von Anforderungen und Architektur über mehrere Abstraktionsstufen hinweg ermöglicht (Silkora Referenz S.31).\\

\subsubsection{Szenario-basierte Ansätze}
Szenario-basierte Ansätze zielen vorrangig darauf ab, die wesentlichen geforderten Interaktionen des Systems mit dessen Umgebung zu definieren und mit den Stakeholdern abzustimmen (Silkora Referenz S.18).
\subsubsection{Kombination der Ansätze}
Wenn die wesentlichen Ziele der Stakeholder bekannt sind, besteht die Möglichkeit, diejenige Architekturalternative auszuwählen, mit der die Ziele am besten erfüllt werden können (Silkora Referenz S.18).
\subsubsection{Bewertung}
