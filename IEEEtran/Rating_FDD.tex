\paragraph{COSMOD-RE (FDD)}
<<<<<<< HEAD
COSMOD-RE eignet sich gut um vier der Probleme zu lösen und teilweise um eines der Probleme zu lösen. Zwei der Probleme werden durch COSMOD-RE nicht gelöst. Bei COSMOD-RE werden sowohl die Architektur als auch die Anforderungen parallel entwickelt. Es findet gegenseitiger Austausch statt, was Kommunikation voraussetzt. Auch bei größeren Distanzen muss trotzdem ein Austausch stattfinden, da ansonsten die parallele Entwicklung von Anforderungen und Software-Architektur nicht möglich ist. Daher bedingt die korrekte Ausführung von COSMOD-RE die Lösung des Problems P1. P2, also konkurrierende Interessen, wird durch den gegenseitigen Austausch ebenfalls gelöst. Bei paralleler Entwicklung und gegenseitigem Austausch müssen Kompromisse gebildet werden, welche die Konkurrierenden Interessen auflösen. Fehlendes Know-How wird durch COSMOD-RE nicht konkret behandelt. Jedoch ist die Auswirkung von fehlendem Know-How bei paralleler Entwicklung und gegenseitigen Austausch eher gering, weswegen das Problem P3 teilweise gelöst wird. COSMOD-RE bietet keine Lösungsansätze gegen einen zu detaillierten und restriktiven Anforderungskatalog mit zu vielen Anforderungen, weswegen dieses Problem P4 offen bleibt. Durch die vier Ebenen in denen bei COSMOD-RE gearbeitet wird und die Untersuchung von Ebenenübergreifenden Beziehungen werden ungenaue Anforderungen verfeinert und fehlende nachgearbeitet weswegen P5 gelöst wird. COSMOD-RE sieht es nicht vor architekturrelevante Anforderungen konkret hervorzuheben weswegen P6 offen bleibt. Durch die Untersuchung auf verschiedenen Ebenen wird P7 gelöst.\\

\paragraph{ADD 3.0 (FDD)}
ADD 3.0 eignet sich gut um eines der Probleme zu lösen und eignet sich nicht dafür vier der Probleme zu lösen. Zwei Probleme werden teilweise gelöst. Da ADD 3.0 eine Methode ist, die die Konzeption der Software-Architektur fördert werden keine der durch den Menschen bedingten Probleme P1, P2 oder P3 behandelt. Dadurch, dass bei jeder Design Round ein Ziel festgelegt wird, was mit dieser Design Round erreicht werden soll werden in jeder Design Round Anforderungen ausgewählt. Dies bedeutet von der großen Menge an restriktiven Anforderungen werden nur einige Zielführende ausgewählt um die aktuelle Design Round durchzuführen. Dadurch wird P4 verhindert. P5 bildet einen Widerspruch zu den Voraussetzungen um ADD 3.0 anzuwenden. Um ADD 3.0 auszuführen müssen die Projekttreiber vorhanden sein, was bedeutet, dass die Anforderungen mithilfe von Szenarien spezifiziert sein sollten. Gleiches gilt für P6. Sind die Projekttreiber nicht wie vorgeschrieben vorbereitet, lässt sich ADD 3.0 nicht anwenden. Es gilt: Garbage in Garbage out \cite{Cer01}. Die in P7 beschriebenen Wechselwirkungen werden nicht weiter beachtet.\\
=======
COSMOD-RE eignet sich gut um vier der Probleme zu l\"osen und teilweise um eines der Probleme zu l\"osen. Zwei der Probleme werden durch COSMOD-RE nicht gel\"ost. Bei COSMOD-RE werden sowohl die Architektur als auch die Anforderungen parallel entwickelt. Es findet gegenseitiger Austausch statt, was Kommunikation voraussetzt. Auch bei gr\"o\ss{}eren Distanz muss trotzdem ein Austausch stattfinden, da ansonsten die parallele Entwicklung von Anforderungen und Software-Architektur nicht m\"oglich ist. Daher bedingt die korrekte Ausf\"uhrung von COSMOD-RE die L\"osung des Problems P1. P2, also konkurrierende Interessen wird durch den gegenseitigen Austausch ebenfalls gel\"ost. Bei paralleler Entwicklung und gegenseitigem Austausch m\"ussen Kompromisse gebildet werden, welche die Konkurrierenden Interessen aufl\"osen. Fehlendes Know-How wird durch COSMOD-RE nicht konkret behandelt. Jedoch ist die Auswirkung von fehlendem Know-How bei paralleler Entwicklung und gegenseitigen Austausch eher gering weswegen, das Problem P3 teilweise gel\"ost wird. COSMOD-RE bietet keine L\"osungsans\"atze gegen einen zu detaillierten und restriktiven Anforderungskatalog mit zu vielen Anforderungen, weswegen dieses Problem P4 offen bleibt. Durch die vier Ebenen in denen bei COSMOD-RE gearbeitet wird und die Untersuchung von Ebenen\"ubergreifenden Beziehungen werden ungenaue Anforderungen verfeinert und fehlende nachgearbeitet weswegen P5 gel\"ost wird. COSMOD-RE sieht es nicht vor architekturrelevante Anforderungen konkret hervorzuheben weswegen P6 offen bleibt. Durch die Untersuchung auf verschiedenen Ebenen wird P7 gel\"ost.\\

\paragraph{ADD 3.0 (FDD)}
ADD 3.0 eignet sich gut um eines der Probleme zu l\"osen und eignet sich nicht daf\"ur vier der Probleme zu l\"osen. Zwei Probleme werden teilweise gel\"ost. Da ADD 3.0 eine Methode ist, die die Konzeption der Software-Architektur f\"ordert werden keine der durch den Menschen bedingten Probleme P1,P2,P3 behandelt. Dadurch, dass bei jeder Design Round ein Ziel festgelegt wird, was mit dieser erreicht werden soll werden in jeder Design Round Anforderungen ausgew\"ahlt. Dies bedeutet von der gro\ss{}en Menge an restriktiven Anforderungen werden nur einige Zielf\"uhrende ausgew\"ahlt um die aktuelle Design Round durchzuf\"uhren. Dadurch wird P4 verhindert. P5 bildet einen Widerspruch zu den Voraussetzungen um ADD 3.0 anzuwenden. Um ADD 3.0 auszuf\"uhren m\"ussen die Projekttreiber vorhanden sein, was bedeutet, dass die Anforderungen mithilfe von Szenarien spezifiziert sein sollten. Gleiches gilt f\"ur P6. Sind die Projekttreiber nicht wie vorgeschrieben vorbereitet, l\"asst sich ADD 3.0 nicht anwenden. Es gilt: \emph{Garbage in Garbage out }\cite{Cer01}. Die in P7 beschriebenen Wechselwirkungen werden nicht weiter beachtet.\\
>>>>>>> 9dae570632d971609558465b4d6bf72e38748fb3
