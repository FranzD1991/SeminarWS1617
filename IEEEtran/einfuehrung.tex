\section{Einf\"uhrung (FDD und JEM)}
Bei der Realisierung eines Softwareentwicklungsprojektes kann eine Kluft zwischen den Software-Architekten auf der einen Seite und den Requirements Engineers auf der anderen Seite bestehen. Software Architekten können mit dem Problem konfrontiert sein, dass Anforderungsdokumente nicht ausreichend sind um weitreichende Architekturentscheidungen in Bezug auf den Softwareentwurf zu treffen. Daher ensteht in diesem Fall für den Software-Architekten häufig der Mehraufwand, dass dieser über weitere Interviews ein präziseres Bild von der gewünschten Software Architektur erhält \cite{probing}. Dies resultiert in Verzögerungen, die sich dann darin wiederspiegeln, dass Termine nicht eingehalten werden können. Betreibt der Software Architekt diesen Mehraufwand nicht und trifft eigene Annahmen bezüglich der Architektur der Software \cite{probing} kann dies wiederum zu einer geringeren Akzeptanz des Kunden und im schlimmsten Fall zum Auftragsverlust führen. Die Requirements Engineers wissen auf der anderen Seite wiederum nicht, welche Anforderungen konkret wichtig für den Software Architekten sind, da ihnen die entsprechende Fachkenntnis fehlt \cite{probing}. Somit können diese nicht zielführend die notwendigen Informationen mit dem Kunden erarbeiten. Ohne diese Informationen ist eine ausreichende Grundlage der Anforderungen für den Architekturentwurf nicht gegeben. Um diese Kluft zu überbrücken ist es notwendig Verfahren zu ermitteln, die es Requirements Engineers ermöglicht die richtigen Informationen einzuholen und die Zusammenarbeit mit den Software Architekten zu optimieren.\\

Im Folgenden werden zunächst vier Verfahren untersucht die das Potenzial haben die Zusammenarbeit zwischen Requirements-Engineer und Software-Architekt zu verbessern. Diese vier Verfahren sind:\\

\begin{itemize}
\item Probing wird beschrieben in \ref{probing}
\item CBSP wird beschrieben in \ref{cbsp}
\item COSMOD-RE wird beschrieben in \ref{scgo}
\item ADD 3.0 wird beschrieben in \ref{add3}\\
\end{itemize}

Diese vier Verfahren werden nach einem festgelegten Schema untersucht. Es werden folgende Punkte betrachtet:\\

\begin{itemize}
\item Ziele
\item Funktionsweise\\
\end{itemize}

Mit den Zielen wird zunächst beschrieben was die Methode bezwecken will. In der Funktionsweise werden folgende Aspekte untersucht:\\

\begin{itemize}
\item Randbedingungen \\
In den Randbedingungen wird geklärt, welche Einschränkungen die Methode hat und welche Vorraussetzungen geklärt sein müssen um die Methodik anzuwenden.
\item Eingabe \\
In der Eingabe wird beschrieben, welche Eingaben die Methodik erhält.
\item Vorgehensmodell \\
Mit dem Vorgehensmodell wird beschrieben wie die Methode anzuwenden ist.
\item Ausgabe \\
In der Ausgabe wird beschrieben, was am Ende das Resultat der Methode ist und welchen Mehrwert dieses haben könnte.\\
\end{itemize}

Nach der Beschreibung der einzelnen Methoden werden diese vergleichend in \ref{vergleich} ausgewertet. Hierfür werden die Methoden zunächst unter den eben genannten Punkten verglichen. Danach werden sie in den Kontext der in \ref{problem} genannten Probleme gebracht und hinsichtlich ihrer Fähigkeit diese zu lösen betrachtet. Es wird überprüft wo diese Verfahren ihre Stärken und Schwächen offenbaren. \\

Abschließend wird überprüft wie diese Verfahren gewinnbringend kombiniert werden oder sich gegenseitig ergänzen können. 