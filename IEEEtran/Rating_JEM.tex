\paragraph{Probing (JEM)}
Da das Probing, wie in Figur \ref{methoden} zu sehen ist, w\"ahrend bzw. vor der Anforderungserhebung angewandt wird, behandelt es die durch den Menschen bedingten Probleme P1, P2, P3 nur teilweise. Es setzt, au\ss{}er den Austausch der Probing Questions (PQs), keine Kommunikation zwischen Software-Architekt und Requirements Engineer voraus oder unterst\"utzt diese. Allerdings erzwingt es mindestens diese eine, den Austausch der PQs. Dadurch verringert das Probing die Distanz zwischen Software-Architekt und Requirements Engineer und  sorgt daf\"ur, dass \"uberhaupt eine Kommunikation stattfindet. P2 behandelt das Probing nur teilweise, da hier zumindest ein gemeinsames Arbeitsmittel, die PQs, von Software-Architekt und Requirements Engineer benutzt wird. Das fehlende Know-How wird hier nur teilweise beim Requirements Engineer durch die PQs aufgebessert, aber nicht beim Software-Architekt. \\

Die Probleme, die mit der Qualit\"at der Anforderungen einhergehen, k\"onnen zur H\"alfte sehr gut und zur H\"alfte gar nicht durch das Probing gel\"ost werden. Probleme P5 und P6 werden ganz explizit durch das Probing behandelt, da es ausschlie\ss{}lich nach architekturrelevanten Anforderungen fragt. Allerdings besch\"aftigt es sich nach der Anforderungserhebung nicht mehr mit den Anforderungen um Probleme mit zu restriktiven oder detaillierten architekturrelevante Anforderungen behandeln zu k\"onnen. Wechselwirkungen zwischen den aufgestellten Anforderungen k\"onnten teilweise aufgedeckt oder ber\"ucksichtigt werden, wenn die Typisierung der PQs betrachtet wird. Jedoch w\"urde dieses Vorgehen nicht vollst\"andig alle Wechselwirkungen aufdecken und ist zudem auch nicht Teil des Modells, sondern w\"are nur eine kleine Hilfestellung. \\

\paragraph{CBSP (JEM)}
CBSP liefert mit Abstand das beste Ergebnis. Da es in den Schritten eins bis vier Kommunikation zwischen Software-Architekt und Requirements Engineer voraussetzt um Konflikte zu l\"osen, behandelt es P1 sehr gut. Die konkurrierenden Interessen aus P2 werden ebenfalls durch die erzwungene Konfliktl\"osung und durch ein festes Vorgehensmodell angegangen. Fehlendes Know-How aus einer der beiden Sparten, Software-Architekt oder Requirements Engineer, wird hierdurch zwar nicht ersetzt, aber die daraus resultierenden Probleme wie sie in \ref{prob_knowhow} aufgef\"uhrt sind werden angegangen. So ist sowohl inhaltlich fehlendes als auch fehlendes Know-How \"uber die Methodik durch das Vorgehensmodell von CBSP irrelevant. Inhaltlich fehlendes Know-How und Missverst\"andnisse bei Fachbegriffen k\"onnen ebenfalls durch die verbesserte Kommunikation angegangen werden. \\

Die Probleme, die mit der Qualit\"at der Anforderungen in Verbindung gebracht werden, werden durch das schrittweise Vorgehen von CBSP bis auf eine Ausnahme vollst\"andig behandelt. Dadurch, dass immer nur eine kleine Menge an Anforderungen pro Iteration identifiziert, verfeinert, auf Abh\"angigkeiten und Redundanzen gepr\"uft werden und alle Anforderungen explizit oder implizit Auswirkungen auf die Architektur haben, wird den Problemen P4, P6 und P7 ausreichend vorgebeugt. Allein P5 wird nur teilweise behandelt, denn durch das iterative Vorgehen und die Beschr\"ankung auf eine Untermenge von Anforderungen pro Iteration k\"onnen zwar ungenaue aber nicht fehlende Anforderungen ber\"ucksichtigt werden. Au\ss{}erdem muss die Spezifikation zu Beginn nicht vollst\"andig sein, was diese M\"oglichkeit auch noch unterst\"utzt. \\


%\begin{itemize}
%\item[P1:] \textit{Schlechte Kommunikation} 
%\item[P2:] \textit{Konkurrierende Interessen}
%\item[P3:] \textit{Fehlendes Know-How} 
%
%\item[P4:] \textit{Zu restriktive / detaillierte architekturrelevante Anforderungen}
%\item[P5:] \textit{Ungenaue/ fehlende architekturrelevante Anforderungen}
%\item[P6:] \textit{Nicht klar hervorgehobene architekturrelevante Anforderungen}
%\item[P7:] \textit{Wechselwirkungen zwischen architekturrelevanten Anforderungen}\\
%\end{itemize}
