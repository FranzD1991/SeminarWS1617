\paragraph{Vergleich zwischen ADD 3.0 und Probing}
Vor und Nachteile im Bezug auf: Parallelität, Probleme, (Zeit, Projektgröße etc.)\\

\paragraph{Vergleich zwischen COSMOD-RE und ADD 3.0}
Im Vergleich von COSMOD-RE und ADD 3.0 gibt es eine Vielzahl von Auffälligkeiten. So haben sowohl ADD 3.0 als auch COSMOD-RE als ein Ziel den Entwurf einer Software-Architektur. Zusätzlich hat COSMOD-RE jedoch noch weitere Ziele, wie die Gewinnung von Anforderungen. Um COSMOD-RE anzuwenden müssen einige Randbedingungen geklärt werden, wie beispielsweise die Definition der Grenzen zwischen verschiedenen Ebenen der Abstraktionshierarchie oder eine Möglichkeit der Verknüpfung von Architektur- und Anforderungsmodellen. Außerdem muss eine System-Vision vorhanden sein. Ähnlich umfangreich sind die Randbedingungen von ADD 3.0. Hier wird vorausgesetzt, dass die Anforderungserhebung abgeschlossen ist und die Projekttreiber vorhanden sind. Besonders die Qualitätsattribute sollten vernünftig erhoben sein. Die ersten großen Unterschiede ergeben sich bei der Betrachtung der Eingaben. COSMOD-RE benötigt als Eingabe lediglich die System-Vision, während ADD 3.0 eine Menge von Projekttreibern benötigt. Die große Menge an ausführlichen Projekttreibern sind ein Nachteil an ADD 3.0, den COSMOD-RE nicht hat. Wenn es darum geht, einen schnellen Einstieg in die Methode zu erlangen, hat COSMOD-RE hier einen Vorteil. Ein Nachteil von COSMOD-RE offenbart sich jedoch, wenn die Ausgaben betrachtet werden. Während COSMOD-RE hier lediglich grobe Architekturartefakte liefert, kann ADD 3.0 eine verfeinerte Software-Architektur generieren. Dies bedeutet wenn der Mehraufwand zu beginn betrieben wird, ist es möglich eine bessere Ausgabe zu produzieren. \\

Besonders auffällig sind die Geschwindigkeiten in denen die Methoden arbeiten. COSMOD-RE hat vier Hierarchieebenen auf denen Anforderungen erhoben werden und die Software-Architektur erstellt wird. Dies wird über die drei CO-Design Prozesse geregelt, in denen jeweils zwei Hierarchieebenen untersucht werden. Jeder dieser drei CO-Design Prozesse führt die fünf Subprozesse aus, die wiederum vergleichbar mit einem Sprint bei SCRUM sind. Somit ist eine Mindestlaufzeit von drei SCRUM-Sprints gegeben, wobei hier nicht berücksichtigt ist, dass jeder der CO-Design Prozesse aufgrund der fünf Subprozesse länger als ein Sprint dauern kann. Bei ADD 3.0 ist ein Durchlauf vergleichbar mit einem Sprint bei SCRUM. Dies bedeutet, die verfeinerte Software-Architektur kann nach einem Sprint bereits fertig sein.\\

ADD 3.0 kann schnell Ergebnisse liefern. Dies ist jedoch hauptsächlich bei kleineren Projekten ein Vorteil. Bei größeren Projekten kann es passieren, dass man sehr unpräzise Ergebnisse erhält, wenn man bei ADD 3.0 nicht genügend Iterationen vollzieht. Hier hat COSMOD-RE den Vorteil, dass hier das Vorgehen durch die verschiedenen Hierarchieebenen und CO-Design Prozesse sehr strukturiert ist.\\

COSMOD-RE beginnt bei der Anforderungsgewinnung und arbeitet bis sowohl die Anforderungen als auch die Software-Architektur erstellt sind. In SP2 wird die Architektursicht erstellt. Es wird jedoch nicht konkret vorgegeben, wie die Architektursicht zu erstellen ist. Dies bedeutet, hier wäre die Freiheit gegeben ADD 3.0 einzusetzen. Dadurch würde COSMOD-RE um ADD 3.0 erweitert werden. Dies würde zudem Lösungsansätze für die zwei Probleme liefern, die COSMOD-RE nicht gelöst bekommt.\\

\paragraph{Vergleich zwischen Probing und  COSMOD-RE}
Vor und Nachteile im Bezug auf: Parallelität, Probleme, (Zeit, Projektgröße etc.)\\