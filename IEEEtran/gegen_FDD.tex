\paragraph{Vergleich zwischen ADD 3.0 und Probing (FDD)}
Vergleicht man ADD 3.0 und Probing f\"allt vor allem auf, dass die Methoden an sehr unterschiedlichen Stellen eingesetzt werden. Probing wird eingesetzt um ASFR zu gewinnen und ADD 3.0 wird eingesetzt um auf der Basis von Anforderungen eine Software-Architektur zu erzeugen. Vergleicht man die Ziele f\"allt dies besonders auf. Bei der Betrachtung der Randbedingungen f\"allt auf, dass hier ebenfalls keine vergleichbare Schnittmenge gegeben ist. Jedoch f\"allt auf, dass eine der Randbedingungen von ADD 3.0 durch Probing gut zu erf\"ullen w\"are. Das Festhalten von architekturrelevanten Anforderungen ist eine gute Grundlage f\"ur die Qualit\"atsattribute, die ADD 3.0 ben\"otigt. Probing lie\ss{}e sich als eine Alternative zu einem Qualit\"ats-Attribut-Workshop ansehen, die es erm\"oglicht mit wom\"oglich geringerem Personalaufwand alle ASFR zu gewinnen. \\

\paragraph{Vergleich zwischen COSMOD-RE und ADD 3.0 (FDD)}
Im Vergleich von COSMOD-RE und ADD 3.0 gibt es eine Vielzahl von Auff\"alligkeiten. So haben sowohl ADD 3.0 als auch COSMOD-RE als ein Ziel den Entwurf einer Software-Architektur. Zus\"atzlich hat COSMOD-RE jedoch noch weitere Ziele, wie die Gewinnung von Anforderungen. Um COSMOD-RE anzuwenden, m\"ussen einige Randbedingungen gekl\"art werden, wie beispielsweise die Definition der Grenzen zwischen verschiedenen Ebenen der Abstraktionshierarchie oder eine M\"oglichkeit der Verkn\"upfung von Architektur- und Anforderungsmodellen. Au\ss{}erdem muss eine System-Vision vorhanden sein. \"Ahnlich umfangreich sind die Randbedingungen von ADD 3.0. Hier wird vorausgesetzt, dass die Anforderungserhebung abgeschlossen ist und die Projekttreiber vorhanden sind. Besonders die Qualit\"atsattribute sollten vern\"unftig erhoben sein. Die ersten gro\ss{}en Unterschiede ergeben sich bei der Betrachtung der Eingaben. COSMOD-RE ben\"otigt als Eingabe lediglich die System-Vision, w\"ahrend ADD 3.0 eine Menge von Projekttreibern ben\"otigt. Die gro\ss{}e Menge an ausf\"uhrlichen Projekttreibern sind ein Nachteil an ADD 3.0, den COSMOD-RE nicht hat. Wenn es darum geht, einen schnellen Einstieg in die Methode zu erlangen, hat COSMOD-RE hier einen Vorteil. Ein Nachteil von COSMOD-RE offenbart sich jedoch, wenn die Ausgaben betrachtet werden. W\"ahrend COSMOD-RE hier lediglich grobe Architekturartefakte liefert, kann ADD 3.0 eine verfeinerte Software-Architektur generieren. Dies bedeutet, wenn der Mehraufwand zu beginn betrieben wird, ist es m\"oglich eine bessere Ausgabe zu produzieren. \\

Besonders auff\"allig sind die Geschwindigkeiten, in denen die Methoden arbeiten. COSMOD-RE hat vier Hierarchieebenen auf denen Anforderungen erhoben werden und die Software-Architektur erstellt wird. Dies wird \"uber die drei CO-Design Prozesse geregelt, in denen jeweils zwei Hierarchieebenen untersucht werden. Jeder dieser drei CO-Design Prozesse f\"uhrt die f\"unf Subprozesse aus, die wiederum vergleichbar mit einem Sprint bei SCRUM sind. Somit ist eine Mindestlaufzeit von drei SCRUM-Sprints gegeben, wobei hier nicht ber\"ucksichtigt ist, dass jeder der CO-Design Prozesse aufgrund der f\"unf Subprozesse l\"anger als ein Sprint dauern kann. Bei ADD 3.0 ist ein Durchlauf vergleichbar mit einem Sprint bei SCRUM. Dies bedeutet, die verfeinerte Software-Architektur kann nach einem Sprint bereits fertig sein.\\

ADD 3.0 kann schnell Ergebnisse liefern. Dies ist jedoch haupts\"achlich bei kleineren Projekten ein Vorteil. Bei gr\"o\ss{}eren Projekten kann es passieren, dass man sehr unpr\"azise Ergebnisse erh\"alt, wenn man bei ADD 3.0 nicht gen\"ugend Iterationen vollzieht. Hier hat COSMOD-RE den Vorteil, dass hier das Vorgehen durch die verschiedenen Hierarchieebenen und CO-Design Prozesse sehr strukturiert ist.\\

COSMOD-RE beginnt bei der Anforderungsgewinnung und arbeitet bis sowohl die Anforderungen als auch die Software-Architektur erstellt sind. In SP2 wird die Architektursicht erstellt. Es wird jedoch nicht konkret vorgegeben, wie die Architektursicht zu erstellen ist. Dies bedeutet, hier w\"are die Freiheit gegeben, ADD 3.0 einzusetzen. Dadurch w\"urde COSMOD-RE um ADD 3.0 erweitert werden. Dies w\"urde zudem L\"osungsans\"atze f\"ur die zwei Probleme liefern, die COSMOD-RE nicht gel\"ost bekommt.\\

\paragraph{Vergleich zwischen Probing und  COSMOD-RE (FDD)}
COSMOD-RE hat als Ziele sowohl Anforderungen zu gewinnen als auch eine Software-Architektur zu erzeugen. Probing hat hingegen die Gewinnung von ASFR als Ziel. Probing soll es erm\"oglichen, schnell alle ASFR zu finden um so eine korrekte Software-Architektur zu erm\"oglichen. W\"ahrenddessen werden bei COSMOD-RE ausgehend von der System-Vision sowohl Anforderungen als auch Architekturartefakte abgeleitet. Betrachtet man die f\"unf Subprozesse bei COSMOD-RE lie\ss{}e sich Probing der Entwicklung der Systemnutzungs-Sicht (SP1) zuordnen. Dies bedeutet bei einer Kombination der Ans\"atze w\"urde durch Probing die initiale Systemnutzung-Sicht gewonnen werden, die dann in weiteren Iterationen verfeinert w\"urde. Eine Kombination der beiden Ans\"atze w\"urde eine L\"osung des Problems P6 f\"ur COSMOD-RE darstellen.\\
