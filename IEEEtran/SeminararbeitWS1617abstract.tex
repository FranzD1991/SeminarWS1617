In der Softwareentwicklung sind das Requirements Engineering und die Software Architektur äußerst wichtige Bestandteile. Ausgehend von einer Menge von Anforderungen wird eine Software-Architektur konzipiert, die dann so implementiert wird. Im Idealfall entspricht die fertiggestellte Softwarelösung dann der System-Vision, die ein Kunde im Kopf hatte als er diese in Auftrag gegeben hat. In der Realität gibt es jedoch an vielen Stellen Probleme, die dazu führen, dass schon die Software Architektur sich deutlich von der System-Vision unterscheidet. Besonders in der Übergangsphase von Anforderungen zu Software-Architektur können starke Unterscheidungen von der System-Vision auftreten. In dieser Arbeit werden Methoden des Requirements Engineering und der Software Architektur Konzeption vorgestellt und verglichen, die das Potenzial haben, dieses Problem zu lösen.