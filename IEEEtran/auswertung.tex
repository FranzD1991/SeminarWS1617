\section{Auswertung der Methoden}\label{auswertung}
In den vorhergehenden Kapiteln sind verschiedene Methoden vorgestellt worden die das Potenzial haben das Zusammenwirken von Requirements Engineer und Software Architekt zu verbessern und dadurch eine verbesserte Software Architektur zu generieren. Nun gilt es jedoch zu prüfen in wie weit dieses Potenzial zutrifft. Eignet sich eine Methode besser als die anderen? Wie sieht es aus wenn die Methoden miteinander verglichen werden? Ergänzen sich die Methoden womöglich? Diese Fragen gilt es in der Auswertung zu beantworten.\\

Hierfür werden zunächst in tabellarischer Form die Kernattribute der besprochenen Methoden dargestellt. Danach werden die Methoden in Bezug zu den Problemen P1-P7 gebracht und es wird mittels eines Ratings überprüft wie geeignet die Methode ist, um das Problem zu bewältigen. Danach werden die Methoden gegenübergestellt, wodurch überprüft werden soll, wo sich die Methoden potenziell ergänzen, bzw. an welchen Stellen die eine Methode besser geeignet ist als die andere.\\ 

\subsection{Vergleich}
In der Tabelle \ref{tab:method_intro} sind die Kernattribute der Methoden dargestellt.\\

\begin{table*}[t]
\caption{Übersicht über die Kernattribute der Methoden}
\centering
\begin{tabular}{|p{1,5cm}|p{3,5cm}|p{3,5cm}|p{3,5cm}|p{3,5cm}|}%{|l|l|l|l|l|}
\hline 
\rule[-1ex]{0pt}{2.5ex} • & \textbf{Probing} &  \textbf{CBSP} & \textbf{COSMOD-RE} & \textbf{ADD 3.0} \\ 
\hline 
\rule[-1ex]{0pt}{2.5ex}  \textbf{Eingabe} &  PQ-Katalog / PQ-Flow & • &  System-Vision & Design Grund / Qualitätsattribute / Primäre Funktionalität / Einschränkungen / Architekturelle Bedenken \\ 
\hline 
\rule[-1ex]{0pt}{2.5ex} \textbf{Ausgabe} & Vollständigere Spezifikation (mit benötigten ASFRs) & • &  System-Nutzung-Ziele / System-Interaktions-Szenarien / grobe Architekturartefakte &  Verfeinerte Software Architektur \\ 
\hline 
\rule[-1ex]{0pt}{2.5ex}  \textbf{Rand-bedingungen} &  Der REler muss gewillt sein, die PQs zu benutzen & • &  Defininition einer Abstraktionshierarchie / Verknüpfung von Anforderungen und Architekturmodellen / Definition von Konsistenzbedinungen / Definition einer System-Vision &  Eingaben müssen gegeben sein / Anforderungserhebung muss abgeschlossen sein / Qualitätsattribute müssen erhoben sein \\ 
\hline 
\rule[-1ex]{0pt}{2.5ex}  \textbf{Ziele} &  Den REler mit PQs ausstatten / Erhebung der ASFRs erleichtern / verbessern & • &  Entwicklung von Anforderungen und Architekturartefakten / Unterstützung der Anordnung von Anforderungen und Architekturartefakten / Definition detaillierter Anforderungen auf der basis von Architekturartefakten &  Konkreter Ansatz zum entwurf einer Software-Architektur / Design einer Software-Architektur \\ 
\hline 
\end{tabular} 
\label{tab:method_intro}
\end{table*}


\subsubsection{Eignung der Methoden zur Lösung der Probleme}
In der Tabelle \ref{tab:method_rating} sind die Methoden und ihr Rating in Bezug zu den Problemen dargestellt.\\

Eine 1 steht hierbei dafür, dass eine Methode nicht geeignet ist, das Problem zu lösen. Die 2 steht dafür, dass eine teilweise dafür geeignet sein kann, das Problem zu lösen und eine 3 steht dafür, dass eine Methode gut dafür geeignet ist, das Problem zu lösen.\\

\begin{figure}[h] %h, t, b, p : here = genau an dieser Stelle, wenn möglich, top = für den Seitenanfang, bottom = für das Seitenende, p = für eine spezielle Seite mit Tabellen und Abbildungen
\caption{Rating der Methoden}
\centering
\begin{tabular}{|c|c|c|c|c|}
\hline 
\rule[-1ex]{0pt}{2.5ex} • & \textbf{Probing} & \textbf{CBSP} & \textbf{COSMOD-RE} & \textbf{ADD 3.0} \\ 
\hline 
\rule[-1ex]{0pt}{2.5ex} \textbf{P1} & 2 & 3 & 3 & 1 \\ 
\hline 
\rule[-1ex]{0pt}{2.5ex} \textbf{P2} & 2 & ? - 3 & 3 & 1 \\ 
\hline 
\rule[-1ex]{0pt}{2.5ex} \textbf{P3} & 2 & ? & 2 & 1 \\ 
\hline 
\rule[-1ex]{0pt}{2.5ex} \textbf{P4} & 1 & ? - 3 & 1 & 3 \\ 
\hline 
\rule[-1ex]{0pt}{2.5ex} \textbf{P5} & 3 & ? - 3 & 3 & 3 \\ 
\hline 
\rule[-1ex]{0pt}{2.5ex} \textbf{P6} & 3 & ? - 3 & 1 & 1 \\ 
\hline 
\rule[-1ex]{0pt}{2.5ex} \textbf{P7} & 1 & 3 & 3 & 1 \\ 
\hline 
\end{tabular} 
\label{tab:method_rating}
\end{figure}


\paragraph{Probing (JEM)}
Da das Probing, wie in Figur \ref{methoden} zu sehen ist, während bzw. vor der Anforderungserhebung angewandt wird, behandelt es die durch den Menschen bedingten Probleme P1, P2, P3 nur teilweise. Es setzt, außer den Austausch der Probing Questions (PQs), keine Kommunikation zwischen Software-Architekt und Requirements Engineer vorraus oder unterstützt diese. Allerdings erzwingt es mindestens diese eine, den Austausch der PQs. Dadurch verringert das Probing die Distanz zwischen Software-Architekt und Requirements Engineer und  sorgt dafür dass überhaupt eine Kommunikation stattfindet. P2 behandelt das Probing nur Teilweise, da hier zumindest ein gemeinsames Arbeitsmittel, die PQs, von Software-Architekt und Requirements Engineer benutzt wird. Das Fehlende Know-How wird hier nur teilweise beim Requirements Engineer durch die PQs aufgebessert, aber nicht beim Software-Architekt. \\
Die Probleme die mit der Qualit\"at der Anforderungen einhergehen können zur Hälfte sehr gut und zur Hälfte gar nicht durch das Probing gelöst werden. Probleme P5 und P6 werden ganz explizit durch das Probing angegangen, da es ausschließlich nach architekturrelevanten Anforderungen fragt. Allerdings beschäftigt es sich nach der Anforderungserhebung nicht mehr mit den Anforderungen um Probleme mit zu restriktive oder detaillierte architekturrelevante Anforderungen behandeln zu können. Wechselwirkungen zwischen den aufgestellten Anforderungen könnten teilweise aufgedeckt oder berücksichtigt werden, wenn die Typisierung der PQs betrachtet wird. Jedoch würde dieses Vorgehen nicht vollständig alle Wechselwirkungen aufdecken und ist zudem auch nicht Teil des Modells, sondern wäre nur eine kleine Hilfestellung. \\

\paragraph{CBSP (JEM)}
Begründung warum die werte in der tabelle so sind wie sie sind. \\
P5 nur Teilweise, alles andere Voll! \\
-> weil bei jeder Iteration immer RQ ausgewählt werden und welche vergessen werden könnten


%\begin{itemize}
%\item[P1:] \textit{Schlechte Kommunikation} 
%\item[P2:] \textit{Konkurrierende Interessen}
%\item[P3:] \textit{Fehlendes Know-How} 
%
%\item[P4:] \textit{Zu restriktive / detaillierte architekturrelevante Anforderungen}
%\item[P5:] \textit{Ungenaue/ fehlende architekturrelevante Anforderungen}
%\item[P6:] \textit{Nicht klar hervorgehobene architekturrelevante Anforderungen}
%\item[P7:] \textit{Wechselwirkungen zwischen architekturrelevanten Anforderungen}\\
%\end{itemize}


\paragraph{COSMOD-RE (FDD)}
COSMOD-RE eignet sich gut um vier der Probleme zu lösen und teilweise um eines der Probleme zu lösen. Zwei der Probleme werden durch COSMOD-RE nicht gelöst. Bei COSMOD-RE werden sowohl die Architektur als auch die Anforderungen parallel entwickelt. Es findet gegenseitiger Austausch statt, was Kommunikation voraussetzt. Auch bei größeren Distanzen muss trotzdem ein Austausch stattfinden, da ansonsten die parallele Entwicklung von Anforderungen und Software-Architektur nicht möglich ist. Daher bedingt die korrekte Ausführung von COSMOD-RE die Lösung des Problems P1. P2, also konkurrierende Interessen, wird durch den gegenseitigen Austausch ebenfalls gelöst. Bei paralleler Entwicklung und gegenseitigem Austausch müssen Kompromisse gebildet werden, welche die Konkurrierenden Interessen auflösen. Fehlendes Know-How wird durch COSMOD-RE nicht konkret behandelt. Jedoch ist die Auswirkung von fehlendem Know-How bei paralleler Entwicklung und gegenseitigen Austausch eher gering, weswegen das Problem P3 teilweise gelöst wird. COSMOD-RE bietet keine Lösungsansätze gegen einen zu detaillierten und restriktiven Anforderungskatalog mit zu vielen Anforderungen, weswegen dieses Problem P4 offen bleibt. Durch die vier Ebenen in denen bei COSMOD-RE gearbeitet wird und die Untersuchung von Ebenenübergreifenden Beziehungen werden ungenaue Anforderungen verfeinert und fehlende nachgearbeitet weswegen P5 gelöst wird. COSMOD-RE sieht es nicht vor architekturrelevante Anforderungen konkret hervorzuheben weswegen P6 offen bleibt. Durch die Untersuchung auf verschiedenen Ebenen wird P7 gelöst.\\

\paragraph{ADD 3.0 (FDD)}
ADD 3.0 eignet sich gut um eines der Probleme zu lösen und eignet sich nicht dafür vier der Probleme zu lösen. Zwei Probleme werden teilweise gelöst. Da ADD 3.0 eine Methode ist, die die Konzeption der Software-Architektur fördert werden keine der durch den Menschen bedingten Probleme P1, P2 oder P3 behandelt. Dadurch, dass bei jeder Design Round ein Ziel festgelegt wird, was mit dieser Design Round erreicht werden soll werden in jeder Design Round Anforderungen ausgewählt. Dies bedeutet von der großen Menge an restriktiven Anforderungen werden nur einige Zielführende ausgewählt um die aktuelle Design Round durchzuführen. Dadurch wird P4 verhindert. P5 bildet einen Widerspruch zu den Voraussetzungen um ADD 3.0 anzuwenden. Um ADD 3.0 auszuführen müssen die Projekttreiber vorhanden sein, was bedeutet, dass die Anforderungen mithilfe von Szenarien spezifiziert sein sollten. Gleiches gilt für P6. Sind die Projekttreiber nicht wie vorgeschrieben vorbereitet, lässt sich ADD 3.0 nicht anwenden. Es gilt: Garbage in Garbage out \cite{Cer01}. Die in P7 beschriebenen Wechselwirkungen werden nicht weiter beachtet.\\


\subsubsection{Gegenüberstellung der Methoden}

\paragraph{Vergleich zwischen CBSP und Probing (JEM)}
Wenn man Eingaben und Ausgaben der beiden Methoden miteinander vergleicht, haben diese keine Schnittmenge. Jedoch f\"allt auf, dass die Ausgabe des Probing der Eingabe von CBSP entspricht. Es ist also m\"oglich diese beiden Methoden zusammen, also hintereinander auszuf\"uhren. Dadurch k\"onnte sogar das Problem P5, welches durch CBSP nur teilweise behandelt wurde vollst\"andig behandelt werden. Auch bei dem Vergleich der Ziele sind keine Redundanzen festzustellen. Beide Methoden setzen an vollkommen verschiedenen Punkten im Entwicklungsprozess, wie in Abbildung \ref{methoden} skizziert wurde, an. Somit sind Probing und CBSP theoretisch kompatibel zueinander. Dies wurde in der Praxis noch nicht \"uberpr\"uft. Hierbei ist allerdings zu beachten, dass Probing sich ausschlie\ss{}lich mit ASFRs besch\"aftigt, wohingegen CBSP auch NFRs mit ber\"ucksichtigt. \\ 

Eine Gemeinsamkeit welche beide Methoden teilen ist, dass sie Vorwissen und Erfahrung beim Software-Architekten voraussetzen, damit die Methoden \"uberhaupt angewandt werden k\"onnen. \\

\paragraph{Vergleich zwischen CBSP und COSMOD-RE (JEM)}
Im Vergleich von CBSP und COSMOD-RE f\"allt auf, dass beide ein \"ahnliches iteratives Vorgehen mit mehreren Schritten oder (Sub-)Prozessen haben. Sonst unterscheiden sich beide Methoden allerdings in vielen Punkten. Der gr\"o\ss{}te Punkt besteht darin, dass COSMOD-RE sowohl Anforderungen als auch die Software-Architektur gleichzeitig in einem Design Prozess generiert, w\"ahrend bei CBSP das CBSP Zwischenmodell sowie die Software-Architektur erstellt werden. Die Anforderungen m\"ussen bei CBSP, wenn auch unvollst\"andig, bereits vorliegen und werden nicht ver\"andert. Ein weiterer gro\ss{}er Unterschied ist die Sichtweise auf die Struktur des gesamten Prozesses.COSMOD-RE f\"uhrt vier Abstraktionsstufen und drei Co-Design Prozesse sowie verschiedene Sichten auf alle Artefakte ein. Wohingegen CBSP sich nach dem Twin-Peaks Modell orientiert und nur eine Zwischenebene bzw. das Zwischenmodell eingef\"uhrt hat, welches f\"ur die Verkn\"upfung von Anforderungs- und Software-Architektur-Artefakten zust\"andig ist. F\"ur diese Verkn\"upfung sorgen allerdings beide Ans\"atze gleicherma\ss{}en. \\
Beim Vergleich der Attribute Eingabe, Ausgabe, Randbedingungen und Ziele, wie in Tabelle \ref{tab:method_intro} zu sehen ist, unterscheiden sich die beiden Methoden ebenfalls. \"Uberschneidungen sind lediglich, dass Architekturartefakte erzeugt werden. Somit kann gesagt werden, dass diese beiden Methoden nicht kompatibel zueinander sind. Hier muss genauer \"uberlegt werden, welche Methode zu einem Projekt besser passt. \\

\paragraph{Vergleich zwischen CBSP und ADD 3.0 (JEM)}
CBSP und ADD 3.0 sind sich sehr \"ahnlich, da Eingaben und Ausgaben beider Methoden nahezu identisch sind. Beide erwarten Anforderungen und produzieren iterativ mit fest definierten Schritten eine Software Architektur. ADD 3.0 ben\"otigt f\"ur sein Vorgehen allerdings noch weitere Informationen, wie einen Design Grund, Einschr\"ankungen und architektonische Bedenken. Auch erwartet es die Qualit\"atsattribute im Gegensatz zu CBSP nicht als Anforderungen, sondern diese m\"ussen als Szenarien abgebildet werden. \\

Der gr\"o\ss{}te Unterschied zwischen den beiden Methoden im Hinblick auf die herausgearbeiteten Probleme ist, dass ADD 3.0 keine durch den Menschen bedingten Probleme behandelt. Es verlangt keine weitere Kommunikation zwischen den beiden Rollen Requirements Engineer und Software-Architekt, wohin gegen dies bei CBSP ein fester Bestandteil ist, um bei Problemen oder Missverst\"andnissen den aktuellen Schritt abschlie\ss{}en zu k\"onnen. \\

Au\ss{}erdem funktioniert CBSP auf Grund der Selektion von Anforderungen pro Iteration auch mit gr\"o\ss{}eren Projekten wohingegen ADD 3.0 besser mit kleineren zurecht kommt. Daf\"ur ist CBSP, vermutlich auch durch die bessere Kommunikation, wesentlich zeitaufw\"andiger. \\

Somit sind diese beiden Methoden nicht kompatibel zu einander. Es muss sich auch hier entschieden werden, welche der beiden Methoden in einem bestimmten Projekt den h\"oheren Nutzen erbringt. \\


\paragraph{Vergleich zwischen ADD 3.0 und Probing}
Vor und Nachteile im Bezug auf: Parallelität, Probleme, (Zeit, Projektgröße etc.)\\

\paragraph{Vergleich zwischen COSMOD-RE und ADD 3.0}
Im Vergleich von COSMOD-RE und ADD 3.0 gibt es eine Vielzahl von Auffälligkeiten. So haben sowohl ADD 3.0 als auch COSMOD-RE als ein Ziel den Entwurf einer Software-Architektur. Zusätzlich hat COSMOD-RE jedoch noch weitere Ziele, wie die Gewinnung von Anforderungen. Um COSMOD-RE anzuwenden müssen einige Randbedingungen geklärt werden, wie beispielsweise die Definition der Grenzen zwischen verschiedenen Ebenen der Abstraktionshierarchie oder eine Möglichkeit der Verknüpfung von Architektur- und Anforderungsmodellen. Außerdem muss eine System-Vision vorhanden sein. Ähnlich umfangreich sind die Randbedingungen von ADD 3.0. Hier wird vorausgesetzt, dass die Anforderungserhebung abgeschlossen ist und die Projekttreiber vorhanden sind. Besonders die Qualitätsattribute sollten vernünftig erhoben sein. Die ersten großen Unterschiede ergeben sich bei der Betrachtung der Eingaben. COSMOD-RE benötigt als Eingabe lediglich die System-Vision, während ADD 3.0 eine Menge von Projekttreibern benötigt. Die große Menge an ausführlichen Projekttreibern sind ein Nachteil an ADD 3.0, den COSMOD-RE nicht hat. Wenn es darum geht, einen schnellen Einstieg in die Methode zu erlangen, hat COSMOD-RE hier einen Vorteil. Ein Nachteil von COSMOD-RE offenbart sich jedoch, wenn die Ausgaben betrachtet werden. Während COSMOD-RE hier lediglich grobe Architekturartefakte liefert, kann ADD 3.0 eine verfeinerte Software-Architektur generieren. Dies bedeutet wenn der Mehraufwand zu beginn betrieben wird, ist es möglich eine bessere Ausgabe zu produzieren. \\

Besonders auffällig sind die Geschwindigkeiten in denen die Methoden arbeiten. COSMOD-RE hat vier Hierarchieebenen auf denen Anforderungen erhoben werden und die Software-Architektur erstellt wird. Dies wird über die drei CO-Design Prozesse geregelt, in denen jeweils zwei Hierarchieebenen untersucht werden. Jeder dieser drei CO-Design Prozesse führt die fünf Subprozesse aus, die wiederum vergleichbar mit einem Sprint bei SCRUM sind. Somit ist eine Mindestlaufzeit von drei SCRUM-Sprints gegeben, wobei hier nicht berücksichtigt ist, dass jeder der CO-Design Prozesse aufgrund der fünf Subprozesse länger als ein Sprint dauern kann. Bei ADD 3.0 ist ein Durchlauf vergleichbar mit einem Sprint bei SCRUM. Dies bedeutet, die verfeinerte Software-Architektur kann nach einem Sprint bereits fertig sein.\\

ADD 3.0 kann schnell Ergebnisse liefern. Dies ist jedoch hauptsächlich bei kleineren Projekten ein Vorteil. Bei größeren Projekten kann es passieren, dass man sehr unpräzise Ergebnisse erhält, wenn man bei ADD 3.0 nicht genügend Iterationen vollzieht. Hier hat COSMOD-RE den Vorteil, dass hier das Vorgehen durch die verschiedenen Hierarchieebenen und CO-Design Prozesse sehr strukturiert ist.\\

COSMOD-RE beginnt bei der Anforderungsgewinnung und arbeitet bis sowohl die Anforderungen als auch die Software-Architektur erstellt sind. In SP2 wird die Architektursicht erstellt. Es wird jedoch nicht konkret vorgegeben, wie die Architektursicht zu erstellen ist. Dies bedeutet, hier wäre die Freiheit gegeben ADD 3.0 einzusetzen. Dadurch würde COSMOD-RE um ADD 3.0 erweitert werden. Dies würde zudem Lösungsansätze für die zwei Probleme liefern, die COSMOD-RE nicht gelöst bekommt.\\

\paragraph{Vergleich zwischen Probing und  COSMOD-RE}
Vor und Nachteile im Bezug auf: Parallelität, Probleme, (Zeit, Projektgröße etc.)\\

\subsection{Bewertung}