\section{Fazit}
In der Untersuchung hat sich gezeigt, dass es eine Vielzahl von Problemen gibt, die Einfluss auf die Kooperation zwischen Requirements Engineer und Software-Architekt haben k\"onnen. Ferner gibt es eine Vielzahl an L\"osungsans\"atzen, die das Potenzial haben, die genannten Probleme zu l\"osen. In der Untersuchung der genannten Methoden hat sich jedoch gezeigt, dass diese geeignet sind um eine Teilmenge der Probleme zu l\"osen. Allerdings haben die Methoden unterschiedliche Schw\"achen, St\"arken sowie Einsatzgebiete in welchen sie sinnvoll Anwendung finden. Zus\"atzlich hat sich gezeigt, dass die verschiedenen Methoden verschiedene Randbedingungen und Voraussetzungen haben, die zu erf\"ullen sind um die Anwendbarkeit zu gew\"ahrleisten. Um alle Probleme zu behandeln ist es jedoch notwendig verschiedene Methoden zu kombinieren. Au\ss{}erden wurde aufgezeigt, dass Erfahrung und Know-How eine signifikante Rolle spielen und ohne diese eine Kooperation zwischen Requirements Engineer und Software-Architekt nicht m\"oglich ist. \\