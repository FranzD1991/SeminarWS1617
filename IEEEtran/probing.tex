\subsection{Probing (JEM)}

\subsubsection{Ziele der Methode}

Das Ziel des Probing ist es, Requirements Engineers mit Fragen auszustatten die eine Erhebung architekturrelevanter Anforderungen, den ASFRs, ermöglichen. Diese Fragen werden Probing Questions (PQ) genannt. \\

Die Haupttreiber architekturrelevanter Entscheidungen sind nicht-funktionale (NFR) bzw. qualitative Anforderungen [9]. Diese haben eine explizite Auswirkung auf die zu erstellende Architektur. Bei funktionalen Anforderungen bzw. ASFRs sind diese Auswirkungen meist implizit und müssen zunächst durch weitere Interviews mit dem Kunden herausgearbeitet werden. Auch sind hier die für einen Software-Architekten erforderlichen Informationen nicht immer klar in der Anforderung aufgeführt [9]. Dies kann zu falschen architektonischen Entscheidungen führen. Um dieses Problem anzugehen beschäftigen sich die Probing Questions mit den ASFRs. Requirements Engineers sollen mit Probing Questions ausgestattet werden um zusätzliche relevante Fragen stellen zu können. Dadurch soll während der Anforderungserhebung eine vollständigere Anforderungsspezifikation erstellt werden, in welcher die ASFRs aussagekräftiger sind [9]. Auch helfen diese den Requirements Engineers ein genaueres Verständnis für den Software-Architekten aufzubauen, welche Informationen er für die Konzeption und Implementierung der Software-Architektur benötigt. \\

\subsubsection{Funktionsweise der Methode}

Funktionsweise der Methode


\paragraph{Randbedingungen}

Randbedingungen


\paragraph{Eingabe}

Eingabe


\paragraph{Vorgehensmodell}

Vorgehensmodell


TODO: PQ Kategorien und Typen aus (8). \\

TODO: PQ-Flow aus (6). \\


\textbf{Paper so far:} \\
(6) Probing for Requirements Knowledge to Simulate Architectural Thinking \\
(8) What you see is what you get: Understanding Architecturally Significant Functional Requirements \\
(9) Identifying Architecturally Significant Functional Requirements \\



\paragraph{Ausgabe}

Ausgabe
