\paragraph{Vergleich zwischen CBSP und Probing (JEM)}
Wenn man Eingaben und Ausgaben der beiden Methoden miteinander vergleicht, haben diese keine Schnittmenge. Jedoch f\"allt auf, dass die Ausgabe des Probing der Eingabe von CBSP entspricht. Es ist also m\"oglich, diese beiden Methoden zusammen, also hintereinander auszuf\"uhren. Dadurch k\"onnte sogar das Problem P5, welches durch CBSP nur teilweise behandelt wurde vollst\"andig behandelt werden. Auch bei dem Vergleich der Ziele sind keine Redundanzen festzustellen. Beide Methoden setzen an vollkommen verschiedenen Punkten im Entwicklungsprozess, wie in Abbildung \ref{methoden} skizziert wurde, an. Somit sind Probing und CBSP theoretisch kompatibel zueinander. Dies wurde in der Praxis noch nicht \"uberpr\"uft. Hierbei ist allerdings zu beachten, dass Probing sich ausschlie\ss{}lich mit ASFRs besch\"aftigt, wohingegen CBSP auch NFRs mit ber\"ucksichtigt. \\ 

Eine Gemeinsamkeit beider Methoden ist, dass sie Vorwissen und Erfahrung beim Software-Architekten voraussetzen, damit die Methoden \"uberhaupt angewandt werden k\"onnen. \\

\paragraph{Vergleich zwischen CBSP und COSMOD-RE (JEM)}
Im Vergleich von CBSP und COSMOD-RE f\"allt auf, dass beide ein \"ahnliches iteratives Vorgehen mit mehreren Schritten oder (Sub-)Prozessen haben. Sonst unterscheiden sich beide Methoden allerdings in vielen Punkten. Der gr\"o\ss{}te Punkt besteht darin, dass COSMOD-RE sowohl Anforderungen als auch die Software-Architektur gleichzeitig in einem Design Prozess generiert, w\"ahrend bei CBSP das CBSP Zwischenmodell sowie die Software-Architektur erstellt werden. Die Anforderungen m\"ussen bei CBSP, wenn auch unvollst\"andig, bereits vorliegen und werden nicht ver\"andert. Ein weiterer gro\ss{}er Unterschied ist die Sichtweise auf die Struktur des gesamten Prozesses. COSMOD-RE f\"uhrt vier Abstraktionsstufen und drei Co-Design Prozesse sowie verschiedene Sichten auf alle Artefakte ein. Wohingegen CBSP sich nach dem Twin-Peaks Modell orientiert und nur eine Zwischenebene bzw. das Zwischenmodell eingef\"uhrt hat, welches f\"ur die Verkn\"upfung von Anforderungs- und Software-Architektur-Artefakten zust\"andig ist. F\"ur diese Verkn\"upfung sorgen allerdings beide Ans\"atze gleicherma\ss{}en. \\
Beim Vergleich der Attribute Eingabe, Ausgabe, Randbedingungen und Ziele, wie in Tabelle \ref{tab:method_intro} zu sehen ist, unterscheiden sich die beiden Methoden ebenfalls. \"Uberschneidungen sind lediglich, dass Architekturartefakte erzeugt werden. Somit kann gesagt werden, dass diese beiden Methoden nicht kompatibel zueinander sind. Hier muss genauer \"uberlegt werden, welche Methode besser f\"ur die Projektdurchf\"uhrung geeignet ist. \\

\paragraph{Vergleich zwischen CBSP und ADD 3.0 (JEM)}
CBSP und ADD 3.0 sind sich sehr \"ahnlich, da Eingaben und Ausgaben beider Methoden nahezu identisch sind. Beide erwarten Anforderungen und produzieren iterativ mit fest definierten Schritten eine Software-Architektur. ADD 3.0 ben\"otigt f\"ur dasq Vorgehen allerdings noch weitere Informationen, wie einen Design Grund, Einschr\"ankungen und architektonische Bedenken. Auch erwartet es die Qualit\"atsattribute im Gegensatz zu CBSP nicht als Anforderungen, sondern diese m\"ussen als Szenarien abgebildet werden. \\

Der gr\"o\ss{}te Unterschied zwischen den beiden Methoden im Hinblick auf die herausgearbeiteten Probleme ist, dass ADD 3.0 keine durch den Menschen bedingten Probleme behandelt. Es verlangt keine weitere Kommunikation zwischen den beiden Rollen Requirements Engineer und Software-Architekt, wohingegen dies bei CBSP ein fester Bestandteil ist, um bei Problemen oder Missverst\"andnissen den aktuellen Schritt abschlie\ss{}en zu k\"onnen. \\

Au\ss{}erdem funktioniert CBSP auf Grund der Selektion von Anforderungen pro Iteration auch mit gr\"o\ss{}eren Projekten, wohingegen ADD 3.0 besser mit kleineren zurecht kommt. Daf\"ur ist CBSP, vermutlich auch durch die bessere Kommunikation, wesentlich zeitaufw\"andiger. \\

Somit sind diese beiden Methoden nicht kompatibel zueinander. Es muss sich auch hier entschieden werden, welche der beiden Methoden in einem bestimmten Projekt den h\"oheren Nutzen erbringt. \\
