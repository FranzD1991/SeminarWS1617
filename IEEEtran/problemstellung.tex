\section{Problemstellung}\label{problem}
W\"ahrend der Zusammenarbeit zwischen Requirements Engineer und Software Architekt k\"onnen vielf\"altige Probleme auftreten. Bei der Untersuchung m\"oglicher Probleme l\"asst sich eine Kategorisierung dieser vornehmen. So sind einige Probleme bedingt durch den Menschen selbst, w\"ahrend andere Probleme sich aus den Anforderungen ergeben. Im folgenden werden die kategorisierten Probleme genauer ausgef\"uhrt. \\

\subsection{Durch den Menschen bedingt}
Bei der Betrachtung der direkt durch den Menschen bedingten Probleme fallen folgende auf:\\

\begin{itemize}
\item[P1:] \textit{Schlechte Kommunikation} 
\item[P2:] \textit{Konkurrierende Interessen}
\item[P3:] \textit{Fehlendes Know-How} \\
\end{itemize}

\subsubsection{Schlechte Kommunikation}
Eine schlechte Kommunikation ist in diesem Zusammenhang als eine Kommunikation zu sehen, bei der wesentliche Intentionen und Informationen nicht \"ubermittelt werden. Ein wichtiger Aspekt schlechter Kommunikation ist die Distanz zwischen den Akteuren innerhalb eines Projektes.\\

Nach \cite{Bja01} kann vor allem Distanz ein Grund f\"ur schlechte Kommunikation sein. Distanzen in der Kommunikation zwischen Requirements Engineer und Software Architekt k\"onnen die Koordination reduzieren, was wiederum Projektverz\"ogerungen hervorrufen kann, sowie das nicht erf\"ullen von Kundenw\"unschen. Distanzen werden nach \cite{Bja01} vor allem als r\"aumliche Distanzen bei global agierenden Projektteams gesehen. Bei r\"aumlichen Distanzen k\"onnen nach \cite{Her01} beispielsweise Begleitprobleme wie das Beherrschen einer Sprache aufkommen. So kann das schlechte Beherrschen einer Sprache zu Missverst\"andnissen f\"uhren und Fehler in den Anforderungen hervorrufen.  Zus\"atzlich k\"onnen nach \cite{Bja01} soziale, kulturelle oder zeitliche Distanzen hinzukommen.\\

Bei nicht ausreichendem Kommunikationsfluss zwischen Requirements Engineer und Software Architekt kann das Problem aufkommen, dass der Requirements Engineer, w\"ahrend der Anforderungsgewinnnung, keine R\"ucksprache mit dem Software Architekten h\"alt \cite{Her01}. Ohne ausreichende R\"ucksprache kann beispielsweise eine Beeintr\"achtigung der Qualit\"at der architekturrelevanten Anforderungen auftreten  \cite{Her01}. Dies kann von fehlenden bis fehlerhaft Anforderungen reichen. Ein weiteres m\"ogliches Problem ist, dass der Software Architekt dem Requirements Engineer nicht ausreichend vermittelt, welche Informationen er f\"ur einen g\"ultigen Architekturentwurf ben\"otigt  \cite{Her01}. Aus diesen Kommunikationsproblemen k\"onnen Missverst\"andnisse in den Verantwortlichkeiten folgen  \cite{Her01}. Auch hier ist eine m\"ogliche Folge die negative Beeinflussung der Qualit\"at der Architekturanforderungen.\\

\subsubsection{Konkurrierende Interessen}
Unter konkurrierenden Interessen ist zu verstehen, dass die Stakeholder des Entwicklungsprozesses der Software-Architektur in der Projektarbeit verschiedene Interessen verfolgen, die miteinander in Konflikt stehen. Die Stakeholder haben zwar eine gemeinsame Vision von dem fertigen Produkt, verfolgen aufgrund variierender Interessen jedoch eine unterschiedliche Art der Zielerreichung \cite{Gru01}. Weiter kann es sein, dass Stakeholder durch ihre unterschiedlichen Sichten auf das Projekt auf verschiedene Arten mit diesem interagieren. Dies kann wiederum mit den Projektzielen konkurrieren. Der Requirements Engineer vertritt hierbei eher die Sicht des Kunden, w\"ahrend der Software Architekt, den Standpunkt des Entwicklers vertritt \cite{DeB01}. Ein Problem kann ebenfalls auftreten, wenn Stakeholder verschiedene Tools, Prozesse und Arbeitsweisen nutzen um das Projektziel zu erreichen. Wenn hier die Kompromissbereitschaft fehlt kann dies die Arbeit erschweren \cite{Her01}. \\

\subsubsection{Fehlendes Know-How}
Fehlendes Know-How kann zu diversen Problemen f\"uhren. Dabei kann hier zwischen zwei Ebenen unterschieden werden. Einerseits k\"onnen Probleme aus fehlendem Know-How \"uber die Methodik der jeweils anderen Rolle entstehen \cite{Gru01}\cite{Her01}. Hierbei k\"onnen unter anderem Probleme bei der Planung und Umsetzung eines Projektes resultieren, da hierdurch potenziell relevante Vorgehensweisen oder Methoden nicht ber\"ucksichtigt werden k\"onnen. Andererseits k\"onnen Probleme aus inhaltlich fehlendem Know-How entstehen \cite{Gru01}. So kann es dem Requirements-Engineer zum Beispiel schwer fallen die Schwerpunkte so zu setzen, dass der Software-Architekt die Informationen erh\"alt, die er ben\"otigt \cite{Ros03}. Dies wird dann zum Problem wenn der Requirements-Engineer nicht wei\ss{} welche Informationen der Software-Architekt f\"ur die Umsetzung ben\"otigt. Ferner k\"onnen Missverst\"andnisse auftreten, wenn der Software-Architekt bei fachspezifischen Begriffen ein anderes Verst\"andnis hat als der Requirements-Engineer \cite{Her01}. \\

\subsection{Qualit\"at der Anforderungen}
Mit der Untersuchung der Probleme, die sich auf die Qualit\"at der Anforderungen beziehen, fallen folgende auf:\\

\begin{itemize}
\item[P4:] \textit{Zu restriktive / detaillierte architekturrelevante Anforderungen}
\item[P5:] \textit{Ungenaue/ fehlende architekturrelevante Anforderungen}
\item[P6:] \textit{Nicht klar hervorgehobene architekturrelevante Anforderungen}
\item[P7:] \textit{Wechselwirkungen zwischen architekturrelevanten Anforderungen}\\
\end{itemize}

\subsubsection{Zu restriktive / detaillierte architekturrelevante Anforderungen}
Vor allem bei sehr gro\ss{}en Projekten kann es passieren, dass es tausende von Anforderungen, die f\"ur die Architektur relevant sind, gibt \cite{Gru01}. Es besteht eine hohe Wahrscheinlichkeit, dass in einem solchen Fall viele Anforderungen Widerspr\"uche erzeugen und den L\"osungsraum zu sehr einschr\"anken. Hier k\"onnen restriktive oder detaillierte Anforderungen die Arbeit von Requirements-Engineer und Software-Architekt unn\"otig einschr\"anken und dadurch erschweren. \\

\subsubsection{Ungenaue / Fehlende architekturrelevante Anforderungen}
In der Anforderungsgewinnung kann es passieren, dass wichtige architekturrelevante Anforderungen nicht erhoben werden, oder wichtige Details fehlen \cite{Ros01}\cite{Ros02}. Dies kann jedoch weitreichende Auswirkungen auf die Software-Architektur haben, da dem Software-Architekt notwendige Informationen bei der Konzeption fehlen. Dieser sieht sich dann gezwungen entweder eine Software-Architektur frei zu entwerfen oder zus\"atzliche kl\"arende Gespr\"ache mit dem Kunden zu f\"uhren. Auch hier kann es passieren, dass bei besonders gro\ss{}en Projekten mit mehreren tausenden Anforderungen, wichtige nicht erhoben werden \cite{Gru01}. Vor allem im Kontext der agilen Softwareentwicklung bedingt ein iteratives Vorgehen, dass manche architekturrelevanten Anforderungen erst w\"ahrend der Modellierung oder sogar Implementierung der Architektur gewonnen werden \cite{Gru01}. \\

\subsubsection{Nicht klar hervorgehobene architekturrelevante Anforderungen}
Eines der Hauptprobleme architekturrelevanter Anforderung liegt darin, dass es \"ublicherweise schwer ist diese zu identifizieren und zu spezifizieren \cite{Gru01}. Werden architekturrelevante Anforderungen f\"ur den Software-Architekten nicht hervorgehoben, kann es passieren, dass dieser wichtige Anforderungen erst sp\"at, wenn \"uberhaupt wahrnimmt \cite{Ros03}. Dies kann zur Folge haben, dass wichtige architekturrelevante Entscheidungen nicht rechtzeitig getroffen werden k\"onnen und zu einem Mehraufwand zu einem sp\"ateren Zeitpunkt f\"uhren \cite{Ros01}. Zus\"atzlich liegt ein Problem in der fehlenden Ausf\"uhrung der Auswirkungen von architekturrelevanten Anforderungen auf die sp\"atere Software-Architektur \cite{Ros03}. So kann hieraus folgen, dass wichtige Projektziele nicht erf\"ullt werden, da der Zeitrahmen des Projektes \"uberschritten wird. \\

\subsubsection{Wechselwirkung zwischen Anforderungen}
Unter Wechselwirkung zwischen Anforderung ist zu verstehen, dass Anforderungen teilweise voneinander abh\"angig sein k\"onnen. Diese Abh\"angigkeit kann auf verschiedene Arten und Weisen erfolgen. Ist beispielsweise ein Ansatz gegeben, bei dem mehrere Abstraktionsstufen vorhanden sind kann es passieren dass wichtige Abh\"angigkeiten, die \"uber mehrere Abstraktionsstufen hinweg reichen nicht ber\"ucksichtigt werden oder weitere Probleme verursachen \cite{Sik01}. \"Anderungen von Anforderungen auf verschiedenen Hierarchieebenen treffen kontinuierlich ein und k\"onnen somit Einfluss auf die Architektur haben und erheblichen \"Anderungsbedarf verursachen \cite{Zor01}. Auch ist es schwierig die Konsistenz und Nachverfolgbarkeit der Projektion von Anforderungen auf die Software-Architektur zu gew\"ahrleisten, da eine Anforderung mehrere architekturrelevante Artefakte betreffen kann. Zus\"atzlich kann ein architekturrelevantes Artefakt mehrere Anforderungen betreffen \cite{Gru01}.\\

Neben den aufgef\"uhrten Problemen gibt es weitere, die hier nicht n\"aher behandelt werden. Darunter f\"allt zum Beispiel ein phasenbezogenes Requirements Engineering.
\\
