 In der Softwareentwicklung sind das Requirements Engineering und der Entwurf einer Software-Architektur \"au\ss{}erst wichtige Bestandteile. Ausgehend von einer Menge von Anforderungen wird eine Software-Architektur konzipiert. Die Softwarel\"osung wird dann entsprechend der Software-Architektur implementiert. Im Idealfall entspricht die fertiggestellte Softwarel\"osung dann der System-Vision des Kunden. In der Realit\"at gibt es jedoch an vielen Stellen Probleme, die dazu f\"uhren, dass schon die Software-Architektur sich deutlich von den Anforderungen oder der System-Vision unterscheidet. Besonders in der Phase, in der aus den Anforderungen die Software-Architektur generiert wird, k\"onnen starke Unterschiede zu der System-Vision auftreten. In dieser Arbeit werden Methoden des Requirements Engineering und der Software-Architektur Konzeption vorgestellt und verglichen, die das Potenzial haben, diese Probleme zu l\"osen.