\paragraph{Probing (JEM)}
Da das Probing, wie in Figur \ref{methoden} zu sehen ist, während bzw. vor der Anforderungserhebung angewandt wird, behandelt es die durch den Menschen bedingten Probleme P1, P2, P3 nur teilweise. Es setzt, außer den Austausch der Probing Questions (PQs), keine Kommunikation zwischen Software-Architekt und Requirements Engineer vorraus oder unterstützt diese. Allerdings erzwingt es mindestens diese eine, den Austausch der PQs. Dadurch verringert das Probing die Distanz zwischen Software-Architekt und Requirements Engineer und  sorgt dafür dass überhaupt eine Kommunikation stattfindet. P2 behandelt das Probing nur Teilweise, da hier zumindest ein gemeinsames Arbeitsmittel, die PQs, von Software-Architekt und Requirements Engineer benutzt wird. Das Fehlende Know-How wird hier nur teilweise beim Requirements Engineer durch die PQs aufgebessert, aber nicht beim Software-Architekt. \\
Die Probleme die mit der Qualit\"at der Anforderungen einhergehen können zur Hälfte sehr gut und zur Hälfte gar nicht durch das Probing gelöst werden. Probleme P5 und P6 werden ganz explizit durch das Probing angegangen, da es ausschließlich nach architekturrelevanten Anforderungen fragt. Allerdings beschäftigt es sich nach der Anforderungserhebung nicht mehr mit den Anforderungen um Probleme mit zu restriktive oder detaillierte architekturrelevante Anforderungen behandeln zu können. Wechselwirkungen zwischen den aufgestellten Anforderungen könnten teilweise aufgedeckt oder berücksichtigt werden, wenn die Typisierung der PQs betrachtet wird. Jedoch würde dieses Vorgehen nicht vollständig alle Wechselwirkungen aufdecken und ist zudem auch nicht Teil des Modells, sondern wäre nur eine kleine Hilfestellung. \\

\paragraph{CBSP (JEM)}
Begründung warum die werte in der tabelle so sind wie sie sind. \\



%\begin{itemize}
%\item[P1:] \textit{Schlechte Kommunikation} 
%\item[P2:] \textit{Konkurrierende Interessen}
%\item[P3:] \textit{Fehlendes Know-How} 
%
%\item[P4:] \textit{Zu restriktive / detaillierte architekturrelevante Anforderungen}
%\item[P5:] \textit{Ungenaue/ fehlende architekturrelevante Anforderungen}
%\item[P6:] \textit{Nicht klar hervorgehobene architekturrelevante Anforderungen}
%\item[P7:] \textit{Wechselwirkungen zwischen architekturrelevanten Anforderungen}\\
%\end{itemize}
