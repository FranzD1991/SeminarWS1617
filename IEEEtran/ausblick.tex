\section{Ausblick}
Im Vergleich der verschiedenen Methoden haben sich theoretische Erkenntnisse ziehen lassen, die in der Praxis noch evaluiert werden müssen. Um dies zu realisieren sind jedoch verschiedene Schritte notwendig. Zunächst muss in der Praxis untersucht werden, ob die hier aufgezeigten Probleme tatsächlich auftreten und wie groß der Einfluss dieser auf die Kooperation von Requirements Engineer und Software-Architekt ist. Danach ist zu testen ob die Methoden in der praktischen Anwendung tatsächlich die Probleme in der ausgeführten Art lösen und den genannten Mehrwert bringen. Desweiteren sind weitere Faktoren wie beispielsweise Projektgröße, Zeit oder weitere Ressourcen zu evaluieren, im Hinblick darauf, wie groß der Einfluss dieser auf Durchführbarkeit und den Erfolg der Methoden ist. Ferner ist zu validieren ob die Methoden sich in der Praxis tatsächlich zusammenführen lassen. Zuletzt ist zu untersuchen ob eine Zusammenführung der Methoden tatsächlich eine Verbesserung herbeiführt. \\

Neben den hier aufgeführten Methoden gibt es weitere wie die in \cite{Sik01} ausgeführte Erweiterung von COSMOD-RE, die das Potenzial haben die gegebenen Probleme zu lösen. Durch einen Vergleich mit weiteren Methoden könnten die Ergebnisse weiter verfeinert und validiert werden. Zusätzlich ist es möglich die einzelnen Methoden um weitere Aspekte zu erweitern und durch die Anwendung in der Praxis zu verbessern. So sind beispielsweise beim Probing nur zu fünf der 15 ASFR Kategorien PQ-Flows ausgearbeitet worden. \\
