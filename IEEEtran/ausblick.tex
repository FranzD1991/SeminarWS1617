\section{Ausblick}
Im Vergleich der verschiedenen Methoden haben sich theoretische Erkenntnisse ziehen lassen, die in der Praxis noch evaluiert werden m\"ussen. Um dies zu realisieren sind jedoch verschiedene Schritte notwendig. Zun\"achst muss in der Praxis untersucht werden, ob die hier aufgezeigten Probleme tats\"achlich auftreten und wie gro\ss{} der Einfluss dieser auf die Kooperation von Requirements Engineer und Software-Architekt ist. Danach ist zu testen ob die Methoden in der praktischen Anwendung tats\"achlich die Probleme in der ausgef\"uhrten Art l\"osen und den genannten Mehrwert bringen. Desweiteren sind weitere Faktoren wie beispielsweise Projektgr\"o\ss{}e, Zeit oder weitere Ressourcen zu evaluieren, im Hinblick darauf, wie gro\ss{} der Einfluss dieser auf Durchf\"uhrbarkeit und den Erfolg der Methoden ist. Ferner ist zu validieren ob die Methoden sich in der Praxis tats\"achlich zusammenf\"uhren lassen. Zuletzt ist zu untersuchen ob eine Zusammenf\"uhrung der Methoden tats\"achlich eine Verbesserung herbeif\"uhrt. \\

Neben den hier aufgef\"uhrten Methoden gibt es weitere wie die in \cite{Sik01} ausgef\"uhrte Erweiterung von COSMOD-RE, die das Potenzial haben die gegebenen Probleme zu l\"osen. Durch einen Vergleich mit weiteren Methoden k\"onnten die Ergebnisse weiter verfeinert und validiert werden. Zus\"atzlich ist es m\"oglich die einzelnen Methoden um weitere Aspekte zu erweitern und durch die Anwendung in der Praxis zu verbessern. So sind beispielsweise beim Probing nur zu f\"unf der 15 ASFR Kategorien PQ-Flows ausgearbeitet worden. \\
