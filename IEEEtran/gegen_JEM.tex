\paragraph{Vergleich zwischen CBSP und Probing (JEM)}
Wenn man Eingaben und Ausgaben der beiden Methoden miteinander vergleicht haben diese keine Schnittmenge. Jedoch fällt auf, dass die Ausgabe des Probing der Eingabe von CBSP entspricht. Es ist also möglich diese beiden Methoden zusammen, also hintereinander auszuführen. Dadurch könnte sogar das Problem P5, welches durch CBSP nur teilweise behandelt wurde vollständig behandelt werden. Auch bei dem Vergleich der Ziele sind keine Redundanzen festzustellen. Beide Methoden setzen an vollkommen verschiedenen Punkten im Entwicklungsprozess, wie in \ref{methoden} skizziert wurde, an. Somit sind Probing und CBSP theoretisch kompatibel zueinander. Dies wurde in der Praxis noch nicht überprüft. Hierbei ist allerdings zu beachten, dass Probing sich ausschließlich mit ASFRs beschäftigt, wohingegen CBSP auch NFRs mit berücksichtigt. \\ 
Eine Gemeinsamkeit welche beide Methoden teilen ist, dass sie Vorwissen und Erfahrung beim Software-Architekten voraussetzen, damit die Methoden überhaupt angewandt werden können. \\

\paragraph{Vergleich zwischen CBSP und COSMOD-RE (JEM)}
Im Vergleich von CBSP und COSMOD-RE fällt auf, dass beide ein ähnliches Iteratives Vorgehen mit mehreren Schritten oder (Sub-)Prozessen haben. Sonst unterscheiden sich beide Methoden allerdings in vielen Punkten. Der größte Punkt besteht darin, dass COSMOD-RE sowohl Anforderungen als auch die Software-Architektur gleichzeitig in einem Design Prozess generiert, während bei CBSP das CBSP Zwischenmodell sowie die Software-Architektur erstellt werden. Die Anforderungen müssen bei CBSP, wenn auch unvollständig, bereits vorliegen und werden nicht verändert. Ein weiterer großer Unterschied ist die Sichtweise auf die Struktur des gesamten Prozesses.COSMOD-RE führt vier Abstraktionsstufen und drei Co-Design Prozesse sowie verschiedene Sichten auf alle Artefakte ein. Wohingegen CBSP sich nach dem Twin-Peaks Modell orientiert und nur eine Zwischenebene bzw. das Zwischenmodell eingeführt hat, welches für die Verknüpfung von Anforderungs- und Software-Architektur-Artefakten zuständig ist. Für diese Verknüpfung sorgen allerdings beide Ansätze gleichermaßen. \\
Beim Vergleich der Attribute Eingabe, Ausgabe, Randbedingungen und Ziele, wie in Tabelle \ref{tab:method_intro} zu sehen ist, unterscheiden sich die beiden Methoden ebenfalls. Überschneidungen sind lediglich, dass Architekturartefakte erzeugt werden. Somit kann gesagt werden, dass diese beiden Methoden nicht Kompatibel zueinander sind. Hier muss genauer überlegt werden, welche Methode zu einem Projekt besser passt. \\

\paragraph{Vergleich zwischen CBSP und ADD 3.0 (JEM)}


ADD 3.0 benötigt wesentlich mehr Input um zu funktionieren

Vor und Nachteile im Bezug auf: Parallelität, Probleme, (Zeit, Projektgröße etc.)\\

vermutlich nicht, da ähnliche Aufgaben (Architektur verfeinern) 

CBSP -> große Projektgröße möglich, Zeitlich aufwendig, \\

TODO: ae, ue, und oe und ß (sz) im Text ersetzen! \\
