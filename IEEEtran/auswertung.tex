\section{Auswertung der Methoden}\label{auswertung}
In den vorhergehenden Kapiteln sind verschiedene Methoden vorgestellt worden die das Potenzial haben das Zusammenwirken von Requirements Engineer und Software Architekt zu verbessern und dadurch eine verbesserte Software Architektur zu generieren. Nun gilt es jedoch zu prüfen in wie weit dieses Potenzial zutrifft. Eignet sich eine Methode besser als die anderen? Wie sieht es aus wenn die Methoden miteinander verglichen werden? Ergänzen sich die Methoden womöglich? Diese Fragen gilt es in der Auswertung zu beantworten.\\

Hierfür werden zunächst in tabellarischer Form die Kernattribute der besprochenen Methoden dargestellt. Danach werden die Methoden in Bezug zu den Problemen P1-P7 gebracht und es wird mittels eines Ratings überprüft wie geeignet die Methode ist, um das Problem zu bewältigen. Danach werden die Methoden gegenübergestellt, wodurch überprüft werden soll, wo sich die Methoden potenziell ergänzen, bzw. an welchen Stellen die eine Methode besser geeignet ist als die andere.\\ 

\subsection{Vergleich}
In der Tabelle \ref{tab:method_intro} sind die Kernattribute der Methoden dargestellt.\\

\begin{table*}[t]
\caption{Übersicht über die Kernattribute der Methoden}
\centering
\begin{tabular}{|p{1,5cm}|p{3,5cm}|p{3,5cm}|p{3,5cm}|p{3,5cm}|}%{|l|l|l|l|l|}
\hline 
\rule[-1ex]{0pt}{2.5ex} • & \textbf{Probing} &  \textbf{CBSP} & \textbf{COSMOD-RE} & \textbf{ADD 3.0} \\ 
\hline 
\rule[-1ex]{0pt}{2.5ex}  \textbf{Eingabe} &  PQ-Katalog / PQ-Flow & • &  System-Vision & Design Grund / Qualitätsattribute / Primäre Funktionalität / Einschränkungen / Architekturelle Bedenken \\ 
\hline 
\rule[-1ex]{0pt}{2.5ex} \textbf{Ausgabe} & Vollständigere Spezifikation (mit benötigten ASFRs) & • &  System-Nutzung-Ziele / System-Interaktions-Szenarien / grobe Architekturartefakte &  Verfeinerte Software Architektur \\ 
\hline 
\rule[-1ex]{0pt}{2.5ex}  \textbf{Rand-bedingungen} &  Der REler muss gewillt sein, die PQs zu benutzen & • &  Defininition einer Abstraktionshierarchie / Verknüpfung von Anforderungen und Architekturmodellen / Definition von Konsistenzbedinungen / Definition einer System-Vision &  Eingaben müssen gegeben sein / Anforderungserhebung muss abgeschlossen sein / Qualitätsattribute müssen erhoben sein \\ 
\hline 
\rule[-1ex]{0pt}{2.5ex}  \textbf{Ziele} &  Den REler mit PQs ausstatten / Erhebung der ASFRs erleichtern / verbessern & • &  Entwicklung von Anforderungen und Architekturartefakten / Unterstützung der Anordnung von Anforderungen und Architekturartefakten / Definition detaillierter Anforderungen auf der basis von Architekturartefakten &  Konkreter Ansatz zum entwurf einer Software-Architektur / Design einer Software-Architektur \\ 
\hline 
\end{tabular} 
\label{tab:method_intro}
\end{table*}


\subsubsection{Eignung der Methoden zur Lösung der Probleme}
In der Tabelle \ref{tab:method_rating} sind die Methoden und ihr Rating in Bezug zu den Problemen dargestellt.\\

Eine 1 steht hierbei dafür, dass eine Methode nicht geeignet ist, das Problem zu lösen. Die 2 steht dafür, dass eine teilweise dafür geeignet sein kann, das Problem zu lösen und eine 3 steht dafür, dass eine Methode gut dafür geeignet ist, das Problem zu lösen.\\

\begin{figure}[h] %h, t, b, p : here = genau an dieser Stelle, wenn möglich, top = für den Seitenanfang, bottom = für das Seitenende, p = für eine spezielle Seite mit Tabellen und Abbildungen
\caption{Rating der Methoden}
\centering
\begin{tabular}{|c|c|c|c|c|}
\hline 
\rule[-1ex]{0pt}{2.5ex} • & \textbf{Probing} & \textbf{CBSP} & \textbf{COSMOD-RE} & \textbf{ADD 3.0} \\ 
\hline 
\rule[-1ex]{0pt}{2.5ex} \textbf{P1} & 2 & 3 & 3 & 1 \\ 
\hline 
\rule[-1ex]{0pt}{2.5ex} \textbf{P2} & 2 & ? - 3 & 3 & 1 \\ 
\hline 
\rule[-1ex]{0pt}{2.5ex} \textbf{P3} & 2 & ? & 2 & 1 \\ 
\hline 
\rule[-1ex]{0pt}{2.5ex} \textbf{P4} & 1 & ? - 3 & 1 & 3 \\ 
\hline 
\rule[-1ex]{0pt}{2.5ex} \textbf{P5} & 3 & ? - 3 & 3 & 2 \\ 
\hline 
\rule[-1ex]{0pt}{2.5ex} \textbf{P6} & 3 & ? - 3 & 1 & 2 \\ 
\hline 
\rule[-1ex]{0pt}{2.5ex} \textbf{P7} & 1 & 3 & 3 & 1 \\ 
\hline 
\end{tabular} 
\label{tab:method_rating}
\end{figure}


\paragraph{Probing}
Begründung warum die werte in der tabelle so sind wie sie sind. \\

\paragraph{CBSP}
Begründung warum die werte in der tabelle so sind wie sie sind. \\


\paragraph{COSMOD-RE}
Begründung warum die werte in der tabelle so sind wie sie sind. \\

\paragraph{ADD 3.0}
Begründung warum die werte in der tabelle so sind wie sie sind. \\


\subsubsection{Gegenüberstellung der Methoden}

\paragraph{Vergleich zwischen CBSP und Probing}
Vor und Nachteile im Bezug auf: Parallelität, Probleme, (Zeit, Projektgröße etc.)\\

\paragraph{Vergleich zwischen CBSP und  COSMOD-RE}
Vor und Nachteile im Bezug auf: Parallelität, Probleme, (Zeit, Projektgröße etc.)\\

\paragraph{Vergleich zwischen CBSP und ADD 3.0}
Vor und Nachteile im Bezug auf: Parallelität, Probleme, (Zeit, Projektgröße etc.)\\

\paragraph{Vergleich zwischen ADD 3.0 und Probing}
Vor und Nachteile im Bezug auf: Parallelität, Probleme, (Zeit, Projektgröße etc.)\\

\paragraph{Vergleich zwischen COSMOD-RE und ADD 3.0}
Im Vergleich von COSMOD-RE und ADD 3.0 gibt es eine Vielzahl von Auffälligkeiten. So haben sowohl ADD 3.0 als auch COSMOD-RE als ein Ziel den Entwurf einer Software-Architektur. Zusätzlich hat COSMOD-RE jedoch noch weitere Ziele, wie die Gewinnung von Anforderungen. Um COSMOD-RE anzuwenden müssen einige Randbedingungen geklärt werden, wie beispielsweise die Definition der Grenzen zwischen verschiedenen Ebenen der Abstraktionshierarchie oder eine Möglichkeit der Verknüpfung von Architektur- und Anforderungsmodellen. Außerdem muss eine System-Vision vorhanden sein. Ähnlich umfangreich sind die Randbedingungen von ADD 3.0. Hier wird vorausgesetzt, dass die Anforderungserhebung abgeschlossen ist und die Projekttreiber vorhanden sind. Besonders die Qualitätsattribute sollten vernünftig erhoben sein. Die ersten großen Unterschiede ergeben sich bei der Betrachtung der Eingaben. COSMOD-RE benötigt als Eingabe lediglich die System-Vision, während ADD 3.0 eine Menge von Projekttreibern benötigt. Die große Menge an ausführlichen Projekttreibern sind ein Nachteil an ADD 3.0, den COSMOD-RE nicht hat. Wenn es darum geht, einen schnellen Einstieg in die Methode zu erlangen, hat COSMOD-RE hier einen Vorteil. Ein Nachteil von COSMOD-RE offenbart sich jedoch, wenn die Ausgaben betrachtet werden. Während COSMOD-RE hier lediglich grobe Architekturartefakte liefert, kann ADD 3.0 eine verfeinerte Software-Architektur generieren. Dies bedeutet wenn der Mehraufwand zu beginn betrieben wird, ist es möglich eine bessere Ausgabe zu produzieren.

Bei COSMOD-RE wird als Eingabe lediglich die Systemvision benötigt.
Vor und Nachteile im Bezug auf: Parallelität, Probleme, (Zeit, Projektgröße etc.)\\

\paragraph{Vergleich zwischen Probing und  COSMOD-RE}
Vor und Nachteile im Bezug auf: Parallelität, Probleme, (Zeit, Projektgröße etc.)\\

\subsection{Bewertung}