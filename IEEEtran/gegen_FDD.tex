\paragraph{Vergleich zwischen ADD 3.0 und Probing}
Vor und Nachteile im Bezug auf: Parallelität, Probleme, (Zeit, Projektgröße etc.)\\

\paragraph{Vergleich zwischen COSMOD-RE und ADD 3.0}
Im Vergleich von COSMOD-RE und ADD 3.0 gibt es eine Vielzahl von Auffälligkeiten. So haben sowohl ADD 3.0 als auch COSMOD-RE als ein Ziel den Entwurf einer Software-Architektur. Zusätzlich hat COSMOD-RE jedoch noch weitere Ziele, wie die Gewinnung von Anforderungen. Um COSMOD-RE anzuwenden müssen einige Randbedingungen geklärt werden, wie beispielsweise die Definition der Grenzen zwischen verschiedenen Ebenen der Abstraktionshierarchie oder eine Möglichkeit der Verknüpfung von Architektur- und Anforderungsmodellen. Außerdem muss eine System-Vision vorhanden sein. Ähnlich umfangreich sind die Randbedingungen von ADD 3.0. Hier wird vorausgesetzt, dass die Anforderungserhebung abgeschlossen ist und die Projekttreiber vorhanden sind. Besonders die Qualitätsattribute sollten vernünftig erhoben sein. Die ersten großen Unterschiede ergeben sich bei der Betrachtung der Eingaben. COSMOD-RE benötigt als Eingabe lediglich die System-Vision, während ADD 3.0 eine Menge von Projekttreibern benötigt. Die große Menge an ausführlichen Projekttreibern sind ein Nachteil an ADD 3.0, den COSMOD-RE nicht hat. Wenn es darum geht, einen schnellen Einstieg in die Methode zu erlangen, hat COSMOD-RE hier einen Vorteil. Ein Nachteil von COSMOD-RE offenbart sich jedoch, wenn die Ausgaben betrachtet werden. Während COSMOD-RE hier lediglich grobe Architekturartefakte liefert, kann ADD 3.0 eine verfeinerte Software-Architektur generieren. Dies bedeutet wenn der Mehraufwand zu beginn betrieben wird, ist es möglich eine bessere Ausgabe zu produzieren.

Bei COSMOD-RE wird als Eingabe lediglich die Systemvision benötigt.
Vor und Nachteile im Bezug auf: Parallelität, Probleme, (Zeit, Projektgröße etc.)\\

\paragraph{Vergleich zwischen Probing und  COSMOD-RE}
Vor und Nachteile im Bezug auf: Parallelität, Probleme, (Zeit, Projektgröße etc.)\\