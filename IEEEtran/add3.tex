\subsection{ADD 3.0 (FDD)}
\subsubsection{Beschreibung}
Attribut-driven-Design (ADD) bezeichnet ein Vorgehensmodell, bei dem iterativ ein Architekturdesign ausgearbeitet wird. ADD wird in Form von sogenannten Design Rounds durchgeführt. Eine Design-Round kann hierbei beispielsweise einem Sprint in SCRUM zugeordnet werden. Dies bedeutet in einem Projekt kann es mehrere Design-Rounds geben, mit denen die Software-Architektur verfeinert wird. \\

Eine Eigenschaft, die bei ADD besonders hervorsticht ist, dass es innerhalb der Design-Rounds eine klare Folge von Anweisungen gibt, die auszuführen sind um die Software-Architektur zu entwickeln. Hierbei ist relevant zu erwähnen, dass in ADD die Dokumentation und Analyse als wichtigste Elemente zur Entwicklung der Software-Architektur betrachtet werden. Nachteil bei ADD ist jedoch, dass die Voraussetzung hierfür ist, dass bereits primäre funktionale Anforderungen und Szenarien erhoben sind. Dies bedeutet ADD findet nicht direkt in der Anforderungsgewinnung Anwendung, sonder erst danach. Insgesamt umfasst ADD sieben Schritte die innerhalb einer Design-Round auszuführen sind.\\

Diese sind:
\begin{itemize}
\item[1:] Review Inputs
\item[2:] Establish Iteration goal by selecting drivers
\item[3:] Choose one or more elements of the system to refine
\item[4:] Choose one or more desing concepts that satisfy the selected drivers
\item[5:] Instantiate architectural elements, allocate responsiblities and define interfaces
\item[6:] Sketch views and record design decisions
\item[7:] Perform analysis of current design and review iteration goal and achievement of desing purpose
\end{itemize}
Um bei ADD eine Design-Round durchführen zu können sind jedoch zunächst einige Eingaben für den Prozess vorzubereiten.\\

Diese sind:
\begin{itemize}
\item Übergeordnete Zielstellung
\item Primäre funktionale Anforderungen
\item Szenarien
\item Einschränkungen
\end{itemize}

\paragraph{Step 1 - Überprüfung der Eingaben}
Zunächst muss sichergestellt werden, dass die übergeordnete Zielstellung für die darauffolgenden Design-Aktivitäten festgelegt ist. Diese kann beispielsweise die erstmalige Erstellung eines Design-Entwurfes oder die Verbesserung eines vorhandenen Architektur-Designs sein. Danach wird überprüft, ob die für die Design-Round relevanten Anforderungen und Szenarien korrekt sind. Hier ist unter anderem zu prüfen ob alle relevanten Stakeholder berücksichtigt werden und ob die erhobenen Anforderungen richtig priorisiert sind. Zuletzt muss noch geprüft werden, ob es Einschränkungen bezüglich der Software-Architektur gibt, die in der Design-Round zu berücksichtigen sind.\\

\paragraph{Step 2 - Festlegung des Ziels der Iteration durch Auswahl von Artefakten}
Eine Design-Runde 