\section{Problemstellung}
Während der Zusammenarbeit zwischen Requirements Engineer und Software Architekt können vielfältige Probleme auftreten. Bei der Untersuchung möglicher Probleme lässt sich eine Kategorisierung dieser vornehmen. So sind einige Probleme \textcolor{red}{bedingt durch das Personal}, während andere Probleme sich aus der Qualität der Anforderungen ergeben. Im folgenden werden die kategorisierten Probleme genauer ausgeführt.

\subsection{Personal}
Bei der Betrachtung der direkt durch das Personal bedingten Probleme fallen folgende besonders auf:
\begin{itemize} 
\item Schlechte Kommunikation
\item Konkurrierende Interessen
\item Fehlendes Know-How
\end{itemize}

\subsubsection{Schlechte Kommunikation}
Das Problem der schlechten Kommunikation lässt sich in zwei Klassen aufteilten. Der ersten Klasse lassen sich Probleme zuordnen, bei denen die Größe des Kommunikationsflusses zwischen Requirements Engineer und Software Architekt nicht ausreichend ist. Der zweiten Klasse werden die Probleme zugeordnet, die aus einem beidseitigen Monolog entstehen. Unter einem beidseitigen Monolog ist hierbei zu verstehen, dass in der Kommunikation zwischen Requirements Engineer und Software Architekt kein richtiger Dialog stattfindet, sondern lediglich Informationen und Handlungsanweisungen ausgetauscht werden. \\

Bei nicht ausreichendem Kommunikationsfluss zwischen Requirements Engineer und Software Architekt kann das Problem aufkommen, dass der Requirements Engineer, während der Anforderungsgewinnnung, keine Rücksprache mit dem Software Architekten hält. Ohne ausreichende Rücksprache kann beispielsweise eine Beeinträchtigung der Qualität der architekturrelevanten Anforderungen auftreten. Dies kann von fehlenden bis fehlerhaft Anforderungen reichen. Der Verursacher dieses Problems ist am ehesten der Requirements Engineer. Ein weiteres mögliches Problem ist, dass der Software Architekt dem Requirements Engineer nicht ausreichend vermittelt, welche Informationen er für einen gültigen Architekturentwurf benötigt. Auch hier ist eine mögliche Folge die negative Beeinflussung der Qualität der Architekturanforderungen.\\

In der zweiten Klasse werden Probleme gruppiert, bei denen die Kommunikation zwischen Requirements Engineer und Software Architekt in Form von unabhängig voneinander stattfindenen Monologen auftritt. Dazu kann es passieren, dass die Gesprächspartner nicht miteinander reden, d.h. aneinander vorbei reden. \\

(V)(;,,,;)(V) WOOP WOOP WOOP TEXT TEXT TEXT